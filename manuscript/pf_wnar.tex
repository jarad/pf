%  template.tex for Biometrics papers
%
%  This file provides a template for Biometrics authors.  Use this
%  template as the starting point for creating your manuscript document.
%  See the file biomsample.tex for an example of a full-blown manuscript.

%  ALWAYS USE THE referee OPTION WITH PAPERS SUBMITTED TO BIOMETRICS!!!
%  You can see what your paper would look like typeset by removing
%  the referee option.  Because the typeset version will be in two
%  columns, however, some of your equations may be too long. DO NOT
%  use the \longequation option discussed in the user guide!!!  This option
%  is reserved ONLY for equations that are impossible to split across
%  multiple lines; e.g., a very wide matrix.  Instead, type your equations
%  so that they stay in one column and are split across several lines,
%  as are almost all equations in the journal.  Use a recent version of the
%  journal as a guide.
%
\documentclass[useAMS,referee,usenatbib]{biom}
%documentclass[useAMS]{biom}
%
%  If your system does not have the AMS fonts version 2.0 installed, then
%  remove the useAMS option.
%
%  useAMS allows you to obtain upright Greek characters.
%  e.g. \umu, \upi etc.  See the section on "Upright Greek characters" in
%  this guide for further information.
%
%  If you are using AMS 2.0 fonts, bold math letters/symbols are available
%  at a larger range of sizes for NFSS release 1 and 2 (using \boldmath or
%  preferably \bmath).
%
%  Other options are described in the user guide. Here are a few:
%
%  -  If you use Patrick Daly's natbib  to cross-reference your
%     bibliography entries, use the usenatbib option
%
%  -  If you use \includegraphics (graphicx package) for importing graphics
%     into your figures, use the usegraphicx option
%
%  If you wish to typeset the paper in Times font (if you do not have the
%  PostScript Type 1 Computer Modern fonts you will need to do this to get
%  smoother fonts in a PDF file) then uncomment the next line
%  \usepackage{Times}

%%%%% PLACE YOUR OWN MACROS HERE %%%%%

\def\bSig\mathbf{\Sigma}
\newcommand{\VS}{V\&S}
\newcommand{\tr}{\mbox{tr}}
\newcommand{\jarad}[1]{{\color{red}JARAD: #1}}
\newcommand{\danny}[1]{{\color{blue}DANNY: #1}}

\usepackage{amsmath}
\usepackage{color}
\usepackage{graphicx}

\graphicspath{{/Users/niemi/Dropbox/SIR_Particle_Filtering/Graphs/}
                       {/Users/niemi/Dropbox/SIR_Particle_Filtering/doesnotexist/}
                       {/Users/Danny/Dropbox/SIR_Particle_Filtering/Graphs/}
                       {/data/home/faculty/meiring/Dropbox/SIR_Particle_Filtering/Graphs/}
                       {./Graphs/}
                       % Add your directories here
}

%  The rotating package allows you to have tables displayed in landscape
%  mode.  The rotating package is NOT included in this distribution, but
%  can be obtained from the CTAN archive.  USE OF LANDSCAPE TABLES IS
%  STRONGLY DISCOURAGED -- create landscape tables only as a last resort if
%  you see no other way to display the information.  If you do do this,
%  then you need the following command.

%\usepackage[figuresright]{rotating}

%%%%%%%%%%%%%%%%%%%%%%%%%%%%%%%%%%%%%%%%%%%%%%%%%%%%%%%%%%%%%%%%%%%%%

%  Here, place your title and author information.  Note that in
%  use of the \author command, you create your own footnotes.  Follow
%  the examples below in creating your author and affiliation information.
%  Also consult a recent issue of the journal for examples of formatting.

\title[Modeling a Disease Epidemic Using Particle Filtering]{Modeling a Disease Epidemic Using Particle Filtering}

%  Here are examples of different configurations of author/affiliation
%  displays.  According to the Biometrics style, in some instances,
%  the convention is to have superscript *, **, etc footnotes to indicate
%  which of multiple email addresses belong to which author.  In this case,
%  use the \email{ } command to produce the emails in the display.

%  In other cases, such as a single author or two authors from
%  different institutions, there should be no footnoting.  Here, use
%  the \emailx{ } command instead.

%  The examples below corrspond to almost every possible configuration
%  of authors and may be used as a guide.  For other configurations, consult
%  a recent issue of the the journal.

%  Single author -- USE \emailx{ } here so that no asterisk footnoting
%  for the email address will be produced.

%\author{John Author\emailx{email@address.edu} \\
%Department of Statistics, University of Warwick, Coventry CV4 7AL, U.K.}

%  Two authors from the same institution, with both emails -- use
%  \email{ } here to produce the asterisk footnoting for each email address

%\author{John Author$^{*}$\email{author@address.edu} and
%Kathy Authoress$^{**}$\email{email2@address.edu} \\
%Department of Statistics, University of Warwick, Coventry CV4 7AL, U.K.}

%  Exactly two authors from different institutions, with both emails
%  USE \emailx{ } here so that no asterisk footnoting for the email address
%  is produced.

%\author
%{John Author\emailx{author@address.edu} \\
%Department of Statistics, University of Warwick, Coventry CV4 7AL, U.K.
%\and
%Kathy Author\emailx{anotherauthor@address.edu} \\
%Department of Biostatistics, University of North Carolina at Chapel Hill,
%Chapel Hill, North Carolina, U.S.A.}

%  Three or more authors from same institution with all emails displayed
%  and footnoted using asterisks -- use \email{ }

%\author{John Author$^*$\email{author@address.edu},
%Jane Author$^{**}$\email{jane@address.edu}, and
%Dick Author$^{***}$\email{dick@address.edu} \\
%Department of Statistics, University of Warwick, Coventry CV4 7AL, U.K}

%  Three or more authors from same institution with one corresponding email
%  displayed

%\author{John Author$^*$\email{author@address.edu},
%Jane Author, and Dick Author \\
%Department of Statistics, University of Warwick, Coventry CV4 7AL, U.K}

%  Three or more authors, with at least two different institutions,
%  more than one email displayed

%\author{John Author$^{1,*}$\email{author@address.edu},
%Kathy Author$^{2,**}$\email{anotherauthor@address.edu}, and
%Wilma Flinstone$^{3,***}$\email{wilma@bedrock.edu} \\
%$^{1}$Department of Statistics, University of Warwick, Coventry CV4 7AL, U.K \\
%$^{2}$Department of Biostatistics, University of North Carolina at
%Chapel Hill, Chapel Hill, North Carolina, U.S.A. \\
%$^{3}$Department of Geology, University of Bedrock, Bedrock, Kansas, U.S.A.}

%  Three or more authors with at least two different institutions and only
%  one email displayed

%\author{John Author$^{1,*}$\email{author@address.edu},
%Wilma Flinstone$^{2}$, and Barney Rubble$^{2}$ \\
%$^{1}$Department of Statistics, University of Warwick, Coventry CV4 7AL, U.K \\
%$^{2}$Department of Geology, University of Bedrock, Bedrock, Kansas, U.S.A.}

\author{Daniel M. Sheinson$^{1,*}$\email{sheinson@pstat.ucsb.edu},
Jarad Niemi$^{2,**}$\email{niemi@iastate.edu}, and
Wendy Meiring$^{3,***}$\email{meiring@pstat.ucsb.edu} \\
$^{1}$Department of Statistics and Applied Probability, University of Califonia--Santa Barbara, \\
 Santa Barbara, CA, U.S.A. \\
$^{2}$Department of Statistics, Iowa State University, Ames, IA, U.S.A. \\
$^{3}$Department of Statistics and Applied Probability, University of Califonia--Santa Barbara, \\
 Santa Barbara, CA, U.S.A.}

\begin{document}

%  This will produce the submission and review information that appears
%  right after the reference section.  Of course, it will be unknown when
%  you submit your paper, so you can either leave this out or put in
%  sample dates (these will have no effect on the fate of your paper in the
%  review process!)

%\date{{\it Received October} 2007. {\it Revised February} 2008.  {\it
%Accepted March} 2008.}

%  These options will count the number of pages and provide volume
%  and date information in the upper left hand corner of the top of the
%  first page as in published papers.  The \pagerange command will only
%  work if you place the command \label{firstpage} near the beginning
%  of the document and \label{lastpage} at the end of the document, as we
%  have done in this template.

%  Again, putting a volume number and date is for your own amusement and
%  has no bearing on what actually happens to your paper!

\pagerange{\pageref{firstpage}--\pageref{lastpage}}
%\volume{64}
%\pubyear{2008}
%\artmonth{December}

%  The \doi command is where the DOI for your paper would be placed should it
%  be published.  Again, if you make one up and stick it here, it means
%  nothing!

%\doi{10.1111/j.1541-0420.2005.00454.x}

%  This label and the label ``lastpage'' are used by the \pagerange
%  command above to give the page range for the article.  You may have
%  to process the document twice to get this to match up with what you
%  expect.  When using the referee option, this will not count the pages
%  with tables and figures.

\label{firstpage}

%  put the summary for your paper here

\begin{abstract}
This paper compares the performance of different particle filtering algorithms (sequential Monte Carlo algorithms) when applied to a stochastic compartmental model of a disease outbreak. As a starting point, we consider a similar model to \citet{skvortsov2012monitoring} which is nonlinear in both the states and the parameters in both the observation and evolution equations, with a multivariate response at each time point. Missing observations are allowed. Motivated by the lack of statistical sophistication shown in \citet{skvortsov2012monitoring}, we review some developments in the particle filtering methodology to account for simultaneous estimation of outbreak intensity and the fixed parameters governing the outbreak. We show how the bootstrap filter used by \citet{skvortsov2012monitoring} degenerates quickly, and that a version of the particle filter developed by \citet{Liu:West:comb:2001} that uses a normal density kernel to resample values of the fixed parameters dramatically reduces this degeneracy and performs more efficiently. We also point out the seemingly uninformative uniform priors used in \citet{skvortsov2012monitoring} were affecting inferences. Finally, we discuss how using multinomial resampling of particles adds extra Monte Carlo variability to the weighted sample approximation of the posterior, as evidenced by wider credible intervals compared with other methods.  % Lastly, we add additional uncertainty to the model and perform estimation using the \citet{Liu:West:comb:2001} particle filter with normal priors on the fixed parameters and stratified resampling.
\end{abstract}

%  Please place your key words in alphabetical order, separated
%  by semicolons, with the first letter of the first word capitalized,
%  and a period at the end of the list.
%

\begin{keywords}
Bayesian estimation; epidemiological modeling; state-space modeling; sequential Monte Carlo; particle filtering.
\end{keywords}

%  As usual, the \maketitle command creates the title and author/affiliations
%  display

\maketitle

%  If you are using the referee option, a new page, numbered page 1, will
%  start after the summary and keywords.  The page numbers thus count the
%  number of pages of your manuscript in the preferred submission style.
%  Remember, ``Normally, regular papers exceeding 25 pages and Reader Reaction
%  papers exceeding 12 pages in (the preferred style) will be returned to
%  the authors without review. The page limit includes acknowledgements,
%  references, and appendices, but not tables and figures. The page count does
%  not include the title page and abstract. A maximum of six (6) tables or
%  figures combined is often required.''

%  You may now place the substance of your manuscript here.  Please use
%  the \section, \subsection, etc commands as described in the user guide.
%  Please use \label and \ref commands to cross-reference sections, equations,
%  tables, figures, etc.
%
%  Please DO NOT attempt to reformat the style of equation numbering!
%  For that matter, please do not attempt to redefine anything!

% Set font size to 10pt
\fontsize{10}{12}\selectfont

\section{Introduction \label{sec:intro}}

% Introduce Bayesian framework, MCMC
In statistical applications where prior knowledge/beliefs is available on unknown quantities, the Bayesian framework is often convenient for performing statistical analysis.  In this case, inference is conducted through the posterior distribution of the unknown quantities given the observed data.  However, the calculation of the posterior distribution can involve complicated integrals which frequently do not have an explicit analytical form.  Advances in computing have helped alleviate this problem through the use of simulation-based methods such as Markov-chain Monte Carlo (MCMC) \citep{Gelf:Smit:samp:1990}, a technique to generate samples from the posterior distribution through simulation from known statistical densities.

% Introduce sequential Monte Carlo
A drawback of MCMC is that when new data become available, the entire analysis needs to be performed again.  That is, samples from the previous analysis cannot be reused and new samples need to be generated, a process that often times requires significant time and computing power.  This is computationally inefficient and impractical in fields where data are collected sequentially and need to be analyzed in real time such as tracking an aircraft using radar or monitoring the stock market.  Sequential Monte Carlo - or particle filtering - methods enable on-line inference by updating the posterior sample as new data become available.  Furthermore, these methods are flexible, general, easy to implement, and amenable to computing in parallel. For a general introduction, please see \cite{Douc:deFr:Gord:sequ:2001} and \cite{cappe2007overview}.

% Introduce bootstrap filter, degeneracy problem
The first and most basic particle filter - named the bootstrap filter - was developed by \citet{Gord:Salm:Smit:nove:1993} and \citet{Kita:mont:1996}.  The bootstrap filter uses an iterative algorithm that samples values (i.e. particles) of the unknown quantities from a prior distribution, propagates the particles forward in time using an evolution equation, calculates weights for each particle using a likelihood function based on the data observed up to that time, and resamples the particles based on these weights.  Issues arise with this algorithm mainly in the resampling step, where particles with large weights emit more copies of themselves into the next iteration of the filter, while particles with small weights get eliminated.  This can lead to \emph{particle attrition}, or a degeneracy of particles toward a single value.  The problem is exacerbated further when one or more of the unknown quantities are held fixed over time, since, in this case, new values are not generated during propagation. Developments in the particle filtering methodology to combat this issue include \citet{Pitt:Shep:filt:1999} and \citet{Liu:West:comb:2001}.

% Introduce APF, KDPF, more sophisiticated approaches
%One way to fight particle degeneracy is by simply increasing the number of particles.  A more efficient way, however, was developed by \citet{Pitt:Shep:filt:1999} called the auxiliary particle filter.  The auxiliary particle filter attempts to avoid degeneracy by looking ahead at the next data point before the resampling step and then samples particles in areas of higher likelihood.  An algorithm developed by \citet{Liu:West:comb:2001} builds on the auxiliary particle filter and distinguishes fixed parameters from unobserved states that evolve over time.  Their version of the particle filter avoids degeneracy by resampling values from a normal kernel density approximation of the posterior distribution of the fixed parameters.  Further improvements to the efficiency of the algorithm include using a measure of nonuniformity of particle weights to determine when to resample \citep{Douc:Capp:Moul:comp:2005}, incorporating an MCMC step to refresh values of the fixed parameters \citep{Gilk:Berz:foll:2001}, and taking advantage of a sufficient statistic structure in the model \citep{Fear:mark:2002}.

% Introduce epidemiological modeling, Skvortsov Math. Bio. paper
Epidemiogical modeling is an area where particle filtering can play an important role.  Models of disease outbreaks can be broken into two parts: a dynamic model of how the epidemic evolves and a measurement model of syndromic data.  Thus, it is easy to conceptualize an epidemic from a Bayesian viewpoint as a filtering problem using a state-space model.  Since many mechanistic models of disease transmission are nonlinear and develop in real time, sequential Monte Carlo is a natural fit. For example, \citet{skvortsov2012monitoring} analyzed a simulated epidemic in the form of a stochastic compartmental epidemiological model, using the bootstrap filter to estimate the state of the epidemic and unknown fixed paremeters.  In this paper, we show estimation efficiency is improved when incorporating more recent developments in the particle filtering methodology.

% Introduce layout of paper (see main document)
The rest of the article proceeds as follows. Section \ref{sec:model} contains a description of state-space models and sequential estimation. Section \ref{sec:filtering} gives a general overview of particle filtering methodologies including the kernel density approach of \cite{Liu:West:comb:2001}.  In Section \ref{sec:apply}, we introduce a nonlinear dynamic model for a disease epidemic similar to the one used in \citet{skvortsov2012monitoring}. In section \ref{sec:results}, we apply the particle filtering methods described in Section \ref{sec:filtering} to the model described in Section \ref{sec:apply}.  %In Section \ref{sec:extend}, we benefit from the efficiency of these particle filtering methods to estimate a more complicated model.  In Section \ref{sec:discussion}, we conclude with some additional particle filtering methods that may be helpful in certain model structures.

\section{State-space models \label{sec:model}}

State-space models are a general class of statistical models used for analysis of dynamic data, including disease outbreaks \citep{Mart:Cone:Lope:Lope:baye:2008,watkins2009disease,merl2009statistical,ludkovski2010optimal,skvortsov2012monitoring}.  State space models are constructed using an observation equation, $y_t \sim p_{y,t}(y_t|x_t,\theta)$, and a state evolution equation, $x_t \sim p_{x,t}(x_t|x_{t-1},\theta)$, where $y_t$ is the observed response, $x_t$ is a latent, dynamic state, and $\theta$ is the unknown fixed parameter, all of which could be vectors. The distributions are assumed known conditional on the values of $\theta$ and $x_t$ in the observation equation and on $\theta$ and $x_{t-1}$ in the evolution equation. Depending on whether the observations and the states are continuous or discrete, the distributions themselves may be continuous or discrete. The distributions are typically assumed to only vary with $x_t$ and $\theta$ and therefore the $t$ subscript on each density is dropped.
%The particle filtering methodology discussed in Sections \ref{sec:filtering} and \ref{sec:discussion} apply equally well when the distributions depend on $t$ so long as they are known.
For simplicity, we also drop the $x$ and $y$ subscript and instead let the arguments make clear which distribution we are referring to. Thus, the state-space model is characterized by $y_t \sim p(y_t|x_t,\theta)$ and $x_t \sim p(x_t|x_{t-1},\theta)$.

Special cases of these state-space models include hidden Markov models \citep{cappe2005inference}, where the state $x_t$ has finite support, and dynamic linear models (DLMs) \citep{West:Harr:baye:1997}, where each distribution is Gaussian with mean a linear function of the states.

\subsection{Sequential estimation}

When data are collected sequentially, it is often of interest to determine the \emph{filtered distribution}, the distribution of the current state and parameters conditional on the data observed up to that time. This distribution describes all of the available information up to time $t$ about the current state of the system and any fixed parameters. It can be updated recursively using Bayes' rule:
\begin{equation}
p(x_t,\theta| y_{1:t}) \propto p(y_t|x_t,\theta)p(x_t,\theta|y_{1:t-1}) \label{eqn:filtered}
\end{equation}
where $y_{1:t} = (y_1,\ldots,y_t)$. Only in special cases can $p(x_t,\theta| y_{1:t})$ be evaluated analytically, e.g. in DLMs when $\theta$ is the observation variance \cite[Sec 4.3,][]{petris2009dynamic}. When analytical tractability is not present, we turn to numerical methods including deterministic versions, e.g. extended Kalman filter and the Gaussian sum filter \citep{Alsp:Sore:nonl:1972}, or Monte Carlo versions such as particle filters.

\section{Particle filtering \label{sec:filtering}}

Particle filtering is a sequential Monte Carlo inferential technique based on sequential use of importance sampling. It aims to approximate equation \eqref{eqn:filtered} through a weighted Monte Carlo realization from this distribution, i.e. $p(x_t,\theta| y_{1:t}) \approx \sum_{j}^J w_t^{(j)} \delta_{(x_t^{(j)},\theta^{(j)})}$ where $w_t^{(j)},j=1,\ldots,J$ are the particle \emph{weights}, $(x_t^{(j)},\theta^{(j)})$ is the particle \emph{location}, and $\delta$ is a Dirac delta function. The following steps provide a rough guideline as to how particle filtering algorithms move from an approximation of $p(x_t,\theta| y_{1:t})$ to that of $p(x_{t+1},\theta| y_{1:t+1})$:

\begin{enumerate}
\item Resample: Sample the $J$ particles with replacement according to their associated weights.
\item Propagate: For each particle $(x_t^{(j)},\theta^{(j)})$, generate a new particle state $x_{t+1}^{(j)}$ by sampling from $p\left(\left. x_{t+1}\right|x_t^{(j)},\theta^{(j)}\right)$.
\item Update: For each new particle $(x_{t+1}^{(j)},\theta^{(j)})$, calculate a new weight $w_{t+1}^{(j)} \propto p\left(\left. y_{t+1}\right|x_{t+1}^{(j)},\theta^{(j)}\right)$.
\end{enumerate}

A variety of techniques have been developed, some of which modify/extend these steps to provide a more efficient approximation to equation \eqref{eqn:filtered} in the sense that with the same computation time a better approximation is achieved. In this paper, we compare three particle filter approaches, namely the bootstrap filter (BF), the auxiliary particle filter (APF) and the kernel density particle filter (KDPF).  For the sake of brevity, we are leaving out further discussion of these particle filters and instead refer the reader to \citet{Gord:Salm:Smit:nove:1993}, \citet{Pitt:Shep:filt:1999}, and \citet{Liu:West:comb:2001}.

%We now introduce three fundamental particle filtering techniques: bootstrap filter, auxiliary particle filter, and the kernel density filter.

%\subsection{Bootstrap filter}
%
%The first successful version of particle filtering is known as the bootstrap filter \citep{Gord:Salm:Smit:nove:1993}. Since this method and the auxiliary particle filter were developed for the situation when $\theta$ is known, we will (for the moment) drop $\theta$ from the notation. Suppose you have a current approximation as in equation \eqref{eqn:approx}. Then for each particle $j$,
%
%\begin{enumerate}
%\item Resample: sample an index $k$ from the set $\{1,\ldots,j,\ldots,J\}$ with associated probabilities $\{w_t^{(1)},\ldots,w_t^{(j)},\ldots,w_t^{(J)}\}$,
%%\item Resample: sample an index $k\in \{1,\ldots,j,\ldots,J\}$ with probability $w_t^{(j)}$,
%\item Propagate: sample $x_{t+1}^{(j)} \sim p\left(\left. x_{t+1}\right|x_t^{(k)}\right)$, and
%\item Calculate weights: $\tilde{w}_{t+1}^{(j)} = p\left(y_{t+1}\left|x_{t+1}^{(j)}\right.\right)$.
%\end{enumerate}
%
%\noindent After weights for all particles have been calculated, these weights need to be renormalized via $w_{t+1}^{(j)} = \tilde{w}_{t+1}^{(j)}\left/ \sum_{l=1}^J \tilde{w}_{t+1}^{(l)} \right.$
%
%This procedure provides an approximation to $p(x_t,\theta| y_{1:t})$ and therefore can be applied recursively provided an initial set of weights $w_0^{(j)}$ and locations $x_0^{(j)}$ for all $j$.  The algorithm as described above uses a multinomial resampling scheme, i.e. when we say `Sample an index $k$ from the set $\{1,\ldots,j,\ldots,J\}$ with associated probabilities $\{w_t^{(1)},\ldots,w_t^{(j)},\ldots,w_t^{(J)}\}$'.  This is not the only way to resample the particles, as will be explained later.
%
%\subsection{Auxiliary particle filter}
%
%A key problem that arises in implementing the bootstrap filter is that $w_t^{(j)}$ may be small for samples $x_t^{(j)}$ that are less likely given $y_t$. These samples will likely be eliminated during resampling, resulting in \emph{particle degeneracy}. One way to fight particle degeneracy is to increase the number of particles, $J$, used in the filter. A more efficient method, however, has been developed by \citet{Pitt:Shep:filt:1999} called the auxiliary particle filter.  The idea behind the auxiliary particle filter is to generate particles with higher likelihood given the data by "looking ahead" at $p(x_{t+1}|x_t^{(j)})$ prior to sampling. Given a weighted random sample of particles at time $t$, the auxiliary particle filter approximates $p(x_{t+1}|y_{1:t+1})$ by the following:
%
%\begin{enumerate}
%\item For each particle $j$, calculate a point estimate of $x_{t+1}^{(j)}$ called $\mu_{t+1}^{(j)}$.  %This is typically taken to be the mean of $p(x_{t+1}|x_t^{(j)})$.
%\item Calculate auxiliary weights and renormalize
%\[ \tilde{g}_{t+1}^{(j)} = w_t^{(j)} p(y_{t+1}|\mu_{t+1}^{(j)}) \qquad g_{t+1}^{(j)} = \tilde{g}_{t+1}^{(j)}\left/\sum_{l=1}^J \tilde{g}_{t+1}^l\right. \]
%\item For each particle $j=1,\ldots,J$
%	\begin{enumerate}
%    \item Sample an index $k$ from the set $\{1,\ldots,j,\ldots,J\}$ with associated probabilities $\{g_{t+1}^{(1)},\ldots,g_{t+1}^{(j)},\ldots,g_{t+1}^{(J)}\}$,
%%	\item Sample an index $k\in\{1,\ldots,j,\ldots,J\}$ with probability $g_{t+1}^{(j)}$.
%	\item Propagate the particle $x_{t+1}^{(j)}\sim p\left(x_{t+1}|x_t^{(k)}\right)$.
%	\item Calculate new weights and renormalize
%\[\tilde{w}_{t+1}^{(j)} = \frac{p\left(y_{t+1}|x_{t+1}^{(j)}\right)}{p\left(y_{t+1}|\mu_{t+1}^{(k)}\right)} \qquad w_{t+1}^{(j)} = \tilde{w}_{t+1}^{(j)}\left/\sum_{l=1}^J \tilde{w}_{t+1}^l \right.\]
%	\end{enumerate}
%\end{enumerate}
%
%The bootstrap filter and the auxiliary particle filter were constructed with the idea that all fixed parameters were known. In order to simultaneously estimate the states and fixed parameters using these methods it is necessary to incorporate the fixed parameters into the state with degenerate evolutions. Due to the resampling step, the number of unique values of the fixed parameters in the particle set will decrease over time, exacerbating the degeneracy problem \citep{Liu:West:comb:2001} \jarad{more references}.
%
%\subsection{Kernel density particle filter} \label{sec:kd}
%
%The particle filter introduced by \cite{Liu:West:comb:2001} is a general way of fighting this degeneracy problem by approximating the set of fixed parameter values by a kernel density estimate and then resampling from this approximation at each step in the filter. This enables us to refresh the distribution of $\theta$ to avoid degeneracy and require a lower number of particles to efficiently run the filter. We will refer to this filter as the kernel density particle filter.
%
%We now reintroduce the fixed parameter $\theta$ into the notation. The kernel density particle filter provides an approximation to the filtered distribution via equation \eqref{eqn:approx}. To make the notation transparent, we introduce subscripts for our fixed parameters, e.g. $\theta_t^{(j)}$ represents the value for $\theta$ at time $t$ in particle $j$. This does not imply that $\theta$ is dynamic, but rather that the particle can have different values for $\theta$ throughout time.
%
%Let $\bar{\theta}_t$ and $V_t$ be the weighted sample mean and weighted sample covariance matrix of the posterior sample $\theta_t^{(1)},\ldots,\theta_t^{(J)}$.  The kernel density particle filter also uses a tuning parameter $\delta$, the discount factor, and two derived quantities $h^2 = 1 - ((3\delta - 1)/2\delta)^2$ and $a^2 = 1 - h^2$ that determine how sharp the kernel density approximation is. Typically $\delta$ is taken to be between 0.95 and 0.99.
%
%With the kernel density particle filter, we move to an approximation of $p(x_{t+1},\theta|y_{1:t+1})$ by the following steps:
%
%\begin{enumerate}
%\item For each particle $j$, calculate a point estimate of $\left(x_{t+1}^{(j)},\theta\right)$ given by $\left(\mu_{t+1}^{(j)},m_t^{(j)}\right)$ where
%    \[
%    \mu_{t+1}^{(j)} = E\left(x_{t+1}\left|x_t^{(j)},\theta_t^{(j)} \right.\right) \qquad
%    m_t^{(j)} = a\theta_t^{(j)} + (1-a)\bar{\theta}_t
%    \]
%\item Calculate auxiliary weights and renormalize
%\[ \tilde{g}_{t+1}^{(j)} = w_t^{(j)} p\left(y_{t+1}\left|\mu_{t+1}^{(j)},m_t^{(j)}\right.\right) \qquad g_{t+1}^{(j)} = \tilde{g}_{t+1}^{(j)}\left/ \sum_{l=1}^J \tilde{g}_{t+1}^l. \right. \]
%\item For each particle $j=1,\ldots,J$
%	\begin{enumerate}
%    \item Sample an index $k$ from the set $\{1,\ldots,j,\ldots,J\}$ with associated probabilities $\{g_{t+1}^{(1)},\ldots,g_{t+1}^{(j)},\ldots,g_{t+1}^{(J)}\}$,
%%	\item Sample an index $k\in\{1,\ldots,j,\ldots,J\}$ with probability $g_{t+1}^{(j)}$.
%	\item Regenerate the fixed parameters
%	\[ \theta_{t+1}^{(j)} \sim N\left( m_t^{(k)}, h^2V_t \right). \]
%	\item Propagate the particle
%	\[ x_{t+1}^{(j)} \sim p\left(x_{t+1}\left|x_t^{(k)},\theta_{t+1}^{(j)}\right.\right). \]
%	\item Calculate weights and renormalize
%	\[\tilde{w}_{t+1}^{(j)} = \frac{p\left(y_{t+1}\left|x_{t+1}^{(j)},\theta_{t+1}^{(j)}\right.\right)}{p\left(y_{t+1}\left|\mu_{t+1}^{(k)},m_t^{(k)}\right.\right)}
%	\qquad
%	w_{t+1}^{(j)} = \tilde{w}_{t+1}^{(j)}\left/\sum_{l=1}^J \tilde{w}_{t+1}^l \right.\]
%	\end{enumerate}
%\end{enumerate}
%
%To use the kernel density particle filter with normal kernels it is necessary to parameterize the fixed parameters so that their support is on the real line. This is not a constraint, but rather a practical implementation detail. We typically use logarithms for parameters that have positive support and the logit function for parameters in the interval (0,1). A parameter $\psi$ bounded on the interval (a,b) can first be rebounded to (0,1) through $(\psi-a)/(b-a)$ and then the logit transformation can be applied.

\subsection{Regeneration of fixed parameters \label{sec:kd}}

In the general algorithm described above, new values of $\theta$ are never sampled. Thus, the distribution of $\theta$ is only altered through elimination of particles during resampling, which can lead to particle degeneracy. The KDPF combats this problem by approximating the set of fixed parameter values by a kernel density estimate and then resampling from this approximation at each iteration of the filter. This enables us to refresh the distribution of $\theta$ to avoid degeneracy and require a lower number of particles to efficiently run the filter.

To use the kernel density particle filter with a kernel that is a mixture of normals, it is necessary to parameterize the fixed parameters so that their support is on the real line. This is not a constraint, but rather a practical implementation detail. We typically use logarithms for parameters that have positive support and the logit function for parameters in the interval (0,1). A parameter $\psi$ bounded on the interval (a,b) can first be rebounded to (0,1) through $(\psi-a)/(b-a)$ and then the logit transformation can be applied. We discuss the sensitivity of the performance of the KDPF to using this kind of transformation later on. %Also, a tuning parameter $\Delta$ - known as the discount factor - is used to govern the variance of the normal kernels (see \citet{Liu:West:comb:2001}). $\Delta$ is typically taken to be between 0.95 and 0.99.

\subsection{Resampling}

Practical implementation of any particle filter involves a choice of resampling function and a decision about when resampling should be performed. The general algorithm described in Section \ref{sec:pf} implicitly uses multinomial resampling. Alternatively, better resampling functions exist including stratified, residual, and systematic resampling. In addition, the frequency of resampling should be minimized to reduce Monte Carlo variability introduced during resampling. Typically a measure of the nonuniformity of particle weights is used, such as effective sample size, entropy, and coefficient of variation \citep{Douc:Capp:Moul:comp:2005}.

\section{Particle filtering in epidemiological models \label{sec:apply}}

We now describe how epidemiological modeling fits into the framework of state-space models with sequential estimation described in Section \ref{sec:model} to set the stage for application of the particle filtering methodology described in Section \ref{sec:filtering}. As in \citet{skvortsov2012monitoring}, we consider a modified SIR model with stochastic fluctuations to describe the dynamics of the disease outbreak \citep{herwaarden1995stochepid, dangerfield2009stochepid, anderson2004sars}.  According to this model, we keep track of the proportion of a population susceptible to ($s$), infected by ($i$), and recovered from ($r$) disease, and we require $s + i + r = 1$.  Letting $s_t$ and $i_t$ be the proportion of the population susceptible to disease and infected by disease at time $t$, respectively, we define $x_t = (i_t,s_t)'$ and follow \citet{skvortsov2012monitoring} to describe the evolution of the epidemic from time $t$ to $t + \tau$ (for $\tau > 0$) by
\begin{equation}
x_{t+\tau}\left|x_t,\theta\right. \sim N_{[0,1]\times[0,1]}\left(f_\tau(x_t,\theta),Q_{\tau}(\theta)\right) \label{eqn:state}
\end{equation}
\noindent where
\[
f_\tau(x_t,\theta) = \left(
\begin{array}{c}
i_t +  \tau\beta i_ts^\nu_t - \tau\gamma i_t \\
s_t - \tau\beta i_ts^{\nu}_t
\end{array}
\right)
\qquad
Q_\tau(\theta) = \left(
\begin{array}{ccccc}
(\beta + \gamma)\tau^2/P^2 & 0 \\
0 & \beta\tau^2/P^2
\end{array}
\right)
\]

%The dynamics of the epidemic with respect to time, $t$, are then described by the following differential equations:

%\begin{align}
%\frac{ds}{dt} &= -\beta is^\nu + \epsilon_\beta \label{eqn:dsdt} \\
%\frac{di}{dt} &= \beta is^\nu - \gamma i - \epsilon_\beta + \epsilon_\gamma \label{eqn:didt}
%\end{align}

%\noindent Here, $\beta$ represents the contact rate of spread of the disease, $\gamma$ is the recovery time, and $\nu$ is the mixing intensity of the population.  $\beta$, $\gamma$, and $\nu$ are each restricted to be non-negative, and $\epsilon_\beta$ and $\epsilon_\gamma$ are uncorrelated, Gaussian random variables with zero mean and standard deviations $\sigma_\beta$ and $\sigma_\gamma$, respectively.  Note that, by definition, $r$ is completely specified once $s$ and $i$ are known.
%
%The standard deviation of the noise terms $\sigma_\beta$ and $\sigma_\gamma$ are approximated by
%\[\sigma_\beta \approx \frac{\sqrt{\beta}}{P} \mbox{, } \sigma_\gamma \approx \frac{\sqrt{\gamma}}{P}\]
%This comes from a well-known scaling rule of random fluctuations in the contact rate and recovery time \citep{ovaskainen2010extinction, herwaarden1995stochepid, dangerfield2009stochepid}.  The evolution of the epidemic from time $t$ to $t + \tau$ (for $\tau > 0$) is then approximated using Euler's method \citep{atkinson1989numericalanalysis}.  Letting $s_t$ and $i_t$ be the proportion of the population susceptible to disease and infected by disease at time $t$, respectively, we have

\noindent Here, $P$ is the population size, $\beta$ represents the contact rate of spread of the disease, $\gamma$ the recovery time, and $\nu$ the mixing intensity of the population.  $\beta$, $\gamma$, and $\nu$ are all non-negative, and we let $\theta = (\beta, \gamma, \nu)'$ be the vector of unknown parameters. $N_{\Omega}(\mu,\Sigma)$ represents the normal distribution with mean $\mu$ and covariance matrix $\Sigma$ truncated onto the set $\Omega$.  Notice that in equation \eqref{eqn:state}, the marginal distributions of $s_t$ and $i_t$ have been truncated onto $[0,1]$ since they are proportions.  To generate values from this density during the "propagate" step of the particle filter, we sample from the untruncated normal distribution, reject any samples outside of $\Omega$, and resample when needed.%As in \citet{skvortsov2012monitoring}, $\xi_s$ and $\xi_i$ are modeled as independent, zero mean Gaussian random variables with variances $\beta\tau^2/P^2$ and $(\beta + \gamma)\tau^2/P^2$, respectively, where $P$ is the population size.  These variances come from a well-known scaling rule of random fluctuations in the contact rate and recovery time \citep{ovaskainen2010extinction, herwaarden1995stochepid, dangerfield2009stochepid}.  %This approximation is more accurate for small $\tau$.

%\subsection{Model}

%When monitoring an epidemic, the true $s_t$ and $i_t$ are unknown at each time $t$.  We only observe data from syndromic surveillance, or the systematic collection and monitoring of public health data by public health agencies \citep{wagner2006biosurveillance, wilson2006synsurveillance}.  These measurements can consist of medical observations such as hospital visits and symptoms as well as non-medical data such as absenteeism from work and queries from search engines or social media \citep{chew2010twitter, schuster2010searchquery, signorini2011twitter, Gins:Mohe:Pate:Bram:Smol:Bril:dete:2009}.  Thus, we can fuse information from syndromic surveillance with the disease transmission dynamics through a state-space model where the observation equation specifies how the observed syndromic data depend on the state of the epidemic and other parameters, while the state equation describes how the epidemic evolves over time.

%First, we specify the state evolution equation.  Let $x_t = (i_t,s_t)'$ represent the state of the epidemic at time $t$, and let $\theta$ represent any unknown, fixed parameters in the model.  Let $\theta = (\beta, \gamma, \nu)'$ and, as in \citet{skvortsov2012monitoring}, we regard the $b_l$'s, $\varsigma_l$'s, and $\sigma_l$'s in \eqref{eqn:dd} as known.  We consider the state equation
%\begin{equation}
%x_{t+\tau}\left|x_t,\theta\right. \sim N_{[0,1]\times[0,1]}\left(f_\tau(x_t,\theta),Q_{\tau}(\theta)\right) \label{eqn:state}
%\end{equation}
%\noindent where
%\[
%f_\tau(x_t,\theta) = \left(
%\begin{array}{c}
%i_t +  \tau\beta i_ts^\nu_t - \tau\gamma i_t \\
%s_t - \tau\beta i_ts^{\nu}_t
%\end{array}
%\right)
%\qquad
%Q_\tau(\theta) = \left(
%\begin{array}{ccccc}
%(\beta + \gamma)\tau^2/P^2 & 0 \\
%0 & \beta\tau^2/P^2
%\end{array}
%\right)
%\]
%\noindent and $N_{\Omega}(\mu,\Sigma)$ represents the normal distribution with mean $\mu$ and covariance matrix $\Sigma$ truncated onto the set $\Omega$.  Notice that in equation \eqref{eqn:state}, the marginal distributions of $s_t$ and $i_t$ have been truncated onto $[0,1]$ since they are proportions whereas it is possible from equation \eqref{eqn:dd} to generate values outside of $[0,1]$.  To generate values from this density during the "propagate" step of the particle filter, we sample from the untruncated normal distribution, reject any samples outside of $\Omega$, and resample when needed.

%A further note on propagating particles is necessary.  Notice from equation \eqref{eqn:state} that the transition density needed to propagate $x_t$ in our model is a multivariate normal density with mean $f_\tau(x_t,\theta)$, and recall that the mean function approximates the disease dynamics better for small $\tau$.  Thus, for data observed at times $t_1$ and $t_2$ where the time between observations, $t_2 - t_1$, is large, the mean of $p(x_{t_2}|x_{t_1}^{(j)})$ for each particle $j$ may be inaccurate if $\tau$ is set to $t_2 - t_1$.  To ensure that the transition density matches the actual disease dynamics more closely, we can choose $d$ to be an integer larger than 1, set $\tau = (t_2 - t_1) / d$, and propagate $x_{t_1}^{(j)}$ forward $d$ times to generate $x_{t_2}^{(j)}$.

When monitoring an epidemic, the true $s_t$ and $i_t$ are unknown at each time $t$. We think of the observed data as positive real numbers related to counts of emergency room visits, prescription sales, or calls to a hotline.  %We only observe data from syndromic surveillance, or the systematic collection and monitoring of public health data by public health agencies \citep{wagner2006biosurveillance, wilson2006synsurveillance}.Next, we describe the observation equation.
We can observe data from these different streams/sources asynchronously in time.  That is, at any time $t$, we can observe data from any subset of the streams (or possibly none of them).  Let $z_{l,t}$ represent data coming from stream $l$ at time $t$, where $l = 1,2,\ldots,L$ and $t = 1,2,\ldots,T$.  %We assume a power law relationship between syndromic observations and the proportion of infected individuals \citep{Gins:Mohe:Pate:Bram:Smol:Bril:dete:2009}, given by $z_{l,t} = b_li_t^{\varsigma_l}$ where $b_l \ge 0$ and $\varsigma_l$ are constants. To incorporate random measurement error, we let $z_{l,t} \sim N(b_li_t^{\varsigma_l},\sigma_l^2)$ where $\sigma_l \ge 0$ is a constant. Since $z_{l,t}$ is constrained to be nonnegative, however,
Following \citet{skvortsov2012monitoring}, we define $y_{l,t} = \log z_{l,t}$ and model the log of the observations by
\begin{equation}
y_{l,t} \sim N\left(\log\left(b_li_t^{\varsigma_l}\right),\sigma_l^2 / (b_li_t^{\varsigma_l})^2\right) \label{eqn:obs}
\end{equation}.
\noindent Then, we define $y_t = (y_{1,t},y_{2,t},\ldots,y_{L,t})'$ and express the likelihood of an observation $y_t$ given $x_t$ and $\theta$ by $p(y_t|x_t,\theta) = N(y_t;\mu_t,\Sigma_t)$ where $\mu_t$ is an $L$-length vector with element $l$ equal to $\log(b_li_t^{\varsigma_l})$ and $\Sigma_t$ is an $L \times L$ diagonal matrix with the $l^{\mbox{th}}$ diagonal equal to $\sigma_l^2 / (b_li_t^{\varsigma_l})^2$.  Elements of $y_t$ may be missing, in which case the dimension of $p(y_t|x_t,\theta)$ shrinks by the number of missing elements. %If the $j^{\mbox{th}}$ element of $y_t$ is missing, then the $j^{\mbox{th}}$ element of $\mu_t$ and $j^{\mbox{th}}$ row and column of $\Sigma_t$ are also missing and thus omitted in the calculation of the likelihood.
If all elements of $y_t$ are empty (i.e. if no syndromic data are observed at time $t$), particle weights remain the same as they were at the previous time step.

Since $p(x_{t+\tau}|x_t,\theta)$ and $p(y_t|x_t,\theta)$ are nonlinear functions of $x_t$ and $\theta$, no closed-form expression for the posterior $p(x_t,\theta|y_{1:t})$ exists.  Thus, we use particle filtering techniques to approximate $p(x_t,\theta|y_{1:t})$ for all $t$.

\section{Results \label{sec:results}}

We now compare the performance of the BF, APF and KDPF using simulated data.  An epidemic lasting $T = 125$ days was simulated from our model for a population of size $P = 5000$.  True values of unknown parameters were set to $\beta = 0.2399$, $\gamma = 0.1066$, and $\nu = 1.2042$, and values of known constants for $L = 4$ streams are the same as those given in Table 1 in \citet{skvortsov2012monitoring}.  Infection was introduced in 10 people in the population at day 0 (i.e. true $i_0 = 10/5000$ and $s_0 = 4990/5000$).  The epidemic evolved over time according to equation \eqref{eqn:state} and data from randomly selected streams at each day were generated from equation \eqref{eqn:obs}.  The simulated epidemic peaks around $t = 50$ days, affecting roughly 17\% of the population.

%\begin{table}[ht]
%\begin{center}
%\caption{Values of known constants in model.}
%\label{tab:true}
%\begin{tabular}{|cccc|}
%\hline
%$l$ & $b_l$ & $\varsigma_l$ & $\sigma_l$ \\
%\hline
%1 & 0.25 & 1.07 & 0.0012 \\
%2 & 0.27 & 1.05 & 0.0008 \\
%3 & 0.23 & 1.01 & 0.0010 \\
%4 & 0.29 & 0.98 & 0.0011 \\
%\hline
%\end{tabular}
%\end{center}
%\end{table}
%
%\begin{figure}
%\centering
%\begin{minipage}{0.48\linewidth}
%\includegraphics[width=1.0\textwidth]{sim-x}
%\end{minipage}
%\begin{minipage}{0.48\linewidth}
%\includegraphics[width=1.0\textwidth]{sim-y}
%\end{minipage}
%\caption{Displays simulated epidemic curves (left) and syndromic observations (right).} \label{fig:data}
%\end{figure}

\subsection{Particle filter runs \label{sec:pf}}

The BF, APF, and KDPF were each run on the simulated data using $J = 100, 1000, 10000, \mbox{ and } 20000$ particles to obtain weighted sample approximations of $p(x_t,\theta|y_{1:t})$ for $t = 1,\ldots,T$. Multinomial, residual, stratified, and systematic resampling were implemented and an effective sample size threshold set at 80\% of the total number of particles was used to determine when to resample particles \citep{Liu:Chen:Wong:reje:1998}. For the KDPF, the discount factor (see \citet{Liu:West:comb:2001}) was set to 0.99 to determine the variance of the normal kernels used to approximate $p(\theta|y_{1:t})$.

$x_0$ and $\theta_0$ were sampled from prior density $p(x_0,\theta) = p(\beta)p(\gamma)p(\nu)p(i_0,s_0)$ where $p(i_0,s_0)$ is the joint pdf of the random variables $i_0$ and $s_0$ with $p(i_0) = N_{[0,1]}(i_0;0.002,0.00255^2)$ and $s_0 = 1 - i_0$. %The prior on $i_0$ and $s_0$ comes from the thought that a very small percentage of the population is infected during the initial stage of an epidemic.  Also, $s_0 = 1 - i_0$ before any infected individuals have recovered from illness, and thus $i_0$ and $s_0$ can be sampled jointly.
To investigate the impact of different prior distributions of $\theta$ on performance of the particle filters, the runs described above were performed once using uniform priors on $\theta$ and then again using log-normal priors.  Bounds for the uniform priors on $p(\beta)$, $p(\gamma)$, and $p(\nu)$ were set at the same as those given in Section 4.2 of \citet{skvortsov2012monitoring}. %\[p(\beta) \sim U(0.14, 0.50) \qquad p(\gamma) = U(0.09, 0.143) \qquad p(\nu) = U(0.95,1.3)\] The uniform bounds given above are the same as those used in \citet{skvortsov2012monitoring}.
Log-normal priors on $\theta$ are given by $\log(\beta) \sim  N(-1.3296, 0.3248^2)$, $\log(\gamma) \sim N(-2.1764, 0.1183^2)$, and $\log(\nu) \sim N(0.1055, 0.0800^2)$. These log-normal priors constrain $\beta$, $\gamma$ and $\nu$ to be positive.  The mean and variance of the log of the unknown parameters were chosen so that random draws on the original scale would fall within the uniform bounds with 95\% probability.

Logit and log transformations were applied to the components of $\theta$ in the manner described at the end of Section \ref{sec:kd} so that the normal kernel could be used in the KDPF while constraining $\beta$, $\gamma$, and $\nu$ to be within their respective prior domains (i.e. logit was used with uniform priors and log with log-normal priors).  To aid comparison of the particle filters, the same prior draws were used in the BF, APF, and KDPF as long as the number of particles and prior were the same.

%\begin{align*}
%\log(\beta) &\sim  N(-1.3296, 0.3248^2) \\
%\log(\gamma) &\sim N(-2.1764, 0.1183^2) \\
%\log(\nu) &\sim N(0.1055, 0.0800^2)
%\end{align*}

%\noindent The normal priors are specified on the log scale so that $\beta$, $\gamma$, and $\nu$ are constrained to be positive, and the mean and variance of the log of the unknown parameters were chosen so that random draws on the original scale would fall within the uniform bounds with 95\% probability.  Logit and log transformations were applied to the components of $\theta$ in the manner described at the end of Section \ref{sec:kd} so that the normal kernel could be used in the KDPF while constraining $\beta$, $\gamma$, and $\nu$ to be within their respective prior domains (i.e. logit was used with uniform priors and log with normal priors).  To aid comparison of the particle filters, the same prior draws were used in the BF, APF, and KDPF as long as the number of particles and prior were the same.

\subsection{Comparison of particle filter algorithms \label{sec:pfcomparison}}

First, we compare the performance of the particle filtering algorithms using uniform priors on $\theta$ and systematic resampling, since these priors and resampling scheme were used in \citet{skvortsov2012monitoring}.  Figure \ref{fig:pfs} shows 95\% credible bounds of the marginal filtered distributions of the unknown parameters over time as $J$ increases.  The bounds for the BF (red lines) and APF (blue lines) start out wide and then degenerate toward a single value because of the elimination of unique particles during resampling.  Although the time of degeneracy increases as $J$ gets larger, the BF and APF bounds become misleading during the second half of the epidemic even for $J = 20000$.  The bounds for the KDPF (green lines), on the other hand, never degenerate because new values of $\theta$ are sampled from the normal density kernel at each iteration of the particle filter.

The KDPF also has an advantage over the BF and APF in terms of computational efficiency.  Notice that the bounds for the KDPF become wider as $J$ increases, but they do not change much between $J = 10000$ and $J = 20000$ particles.  This suggests that by 10000 particles the weighted sample approximation of $p(x_t,\theta|y_{1:t})$ has converged to the true posterior over the entire epidemic period, unlike with the BF and APF.

\begin{figure}
\centering
\includegraphics[width=1.0\textwidth]{PF-systematic-uniform}
\caption{2.5\%/97.5\% quantiles of the filtered distributions over time of the unknown parameters (columns) for increasing number of particles (rows).  Uniform priors and systematic resampling were used throughout.  The axes are identical within each column.  True parameter values are indicated by black horizontal lines.} \label{fig:pfs}
\end{figure}

\subsection{Comparison of priors}

Next, we investigate the sensitivity of the KDPF with $J = 10000$ to using log-normal versus uniform priors on $\theta$.  Figure \ref{fig:priors} compares scatterplots of $\beta$ versus $\gamma$ taken from their joint filtered distributions at $t = 15, 30, 45, 60$ for different prior distributions and transformations used on $\theta$.  The first row illustrates that uniform priors with the logit transformation cause some particles for $\gamma$ to concentrate close to their bounds, especially noticeable for $t \in \{15, 30\}$.  This truncation of points indicates that the uniform bounds on $\gamma$ are too restrictive to account for the uncertainty in the recovery time during the climb of the epidemic.%  This problem does not present itself with $\beta$ because more people are getting sick than recovering from illness during the first half of the epidemic.  Thus, we learn more about the rate of spread of disease than the recovery time and the range of $\beta$ values moves well within the uniform bounds rather quickly.

We also see from the first row of Figure \ref{fig:priors} that $\beta$ and $\gamma$ become correlated during the climb of the epidemic and points begin to form an S-shape at $t = 30$.  This is a product of the logit transformation extending values of $\gamma$ close to the boundary to the whole real line.  To relax the restriction on $\gamma$ imposed by the uniform bounds, we ran the particle filter again using the same prior draws as in the first row of Figure \ref{fig:priors}, but without any transformation on $\theta$.  Scatterplots of $\beta$ versus $\gamma$ for this run are shown in the second row of Figure \ref{fig:priors}, and we see now that particles at $t = 15$ and $t = 30$ that would have been squeezed against the boundary by the logit transformation now lay outside of the uniform bounds for $\gamma$.  There is no longer an S-shape in any of the graphs, and also noticeable is that the sample correlation coefficient between $\beta$ and $\gamma$ at $t = 60$ is larger than when the logit transformation was applied to $\theta$.  This suggests that the more drastic decrease in correlation between $t = 45$ and $t = 60$ in the first row is not driven by the data, but rather is an artifact of the restrictive uniform bounds enforced by the logit transformation.

Ideally, we'd like to apply some transformation to extend $\theta$ to the whole real line so that the KDPF operates smoothly, but we don't want the transformation to influence the data in any way.  The third row of the figure shows the KDPF run using log-normal priors with a log transformation on the components of $\theta$.  As in the second row of Figure \ref{fig:priors}, the distribution of points do not appear truncated or S-shaped.  While a majority of the points in the third row of Figure \ref{fig:priors} are within the uniform bounds, some lay outside because the transformation is not overly restrictive.  We prefer this choice of prior distribution and transformation because it allows us to use prior knowledge of the epidemic to encourage points to lie between the bounds, but is flexible in the event of model mis-specification and allows the data to dictate the evolution of $p(x_t,\theta|y_{1:t})$.

\begin{figure}

\begin{minipage}{1.0\linewidth}
\includegraphics[width=1.0\textwidth]{Hist-KD-uniform-systematic-10000-betagamma}
\vspace{-1.0cm}
\begin{center}
uniform prior draws, logit transformation
\end{center}
\end{minipage}

\vspace{0.5cm}

\begin{minipage}{1.0\linewidth}
\includegraphics[width=1.0\textwidth]{Hist-KD-semi-uniform-systematic-10000-betagamma}
\vspace{-1.0cm}
\begin{center}
uniform prior draws, no transformation
\end{center}
\end{minipage}

\vspace{0.5cm}

\begin{minipage}{1.0\linewidth}
\includegraphics[width=1.0\textwidth]{Hist-KD-normal-systematic-10000-betagamma}
\vspace{-1.0cm}
\begin{center}
normal prior draws, log transformation
\end{center}
\end{minipage}

\caption{Displays scatterplots of $\beta$ (horizontal) versus $\gamma$ (vertical) at $t = 15, 30, 45, 60$ days using the KDPF with $J = 10000$ particles and systematic resampling.  Each row corresponds to a specific prior and transformation applied to $\theta$ in the KDPF.  For simplicity, in each panel only 500 particles sampled from the weighted sample approximation of $p(x_t,\theta|y_{1:t})$ are shown.  The sample correlation coefficient, $r$, displayed in the subtitle of each plot, is computed from the original particle sample.  Red crosses indicate the true values of $\beta$ and $\gamma$ used for simulation ($\beta = 0.2399$ and $\gamma = 0.1066$).  Axes are the same in each panel.  Dashed horizontal lines indicate the bounds of the uniform prior on $\gamma$.  Uniform bounds on $\beta$ are not shown because they lay outside the range of the x-axis.} \label{fig:priors}

\end{figure}

\subsection{Comparison of resampling schemes}

We turn to a comparison of different techniques for the resampling step of the KDPF with 10000 particles and log-normal priors.  Resampling is meant to rebalance the weights of the particles in order to avoid degeneracy.  The four resampling methods mentioned in Section  \ref{sec:pf} achieve this at the cost of increasing the Monte Carlo variability of the particle sample.  That is, additional variance is introduced into the weighted sample approximation of $p(x_t,\theta|y_{1:t})$ due to resampling of particles.

\citet{Douc:Capp:Moul:comp:2005} explains these four methods in detail and determines that 1) multinomial resampling introduces more Monte Carlo variability than does residual or stratified resampling and 2) systematic resampling may introduce more or less Monte Carlo variability than does multinomial resampling. Our analysis (results not shown) shows that - for this specific model and data set - multinomial resampling is outperformed by the other methods in the sense that the marginal filtered distribution of $\theta$ approaches the true posterior the fastest with increasing number of particles. We prefer to use stratified resampling since, according to \cite{Douc:Capp:Moul:comp:2005}, this technique is guaranteed to decrease the Monte Carlo variability in the particle samples, on average, relative to multinomial sampling.

%\begin{enumerate}
%\item Multinomial resampling introduces more Monte Carlo variability than do residual and stratified resampling.
%\item Residual and stratified resampling introduce the same amount of Monte Carlo variability on average.
%\item Systematic resampling may introduce more or less Monte Carlo variability than does multinomial resampling.
%\end{enumerate}

%Figure \ref{fig:resamp} shows 95\% credible bounds of the marginal filtered distributions of the unknown parameters as $J$ increases for the four different resampling schemes.  Ideally, the resampling scheme that performs best would stand out in that the marginal filtered distribution of $\theta$ approaches the true posterior (approximated by the white area in the plots) the fastest with increasing $J$.  Noticing in particular the panel for $\nu$ at $J = 10000$, multinomial resampling appears to be outperformed by the other three methods.  However, it is difficult to discern which of the remaining three methods performs the best.  We prefer to use stratified resampling since, according to \cite{Douc:Capp:Moul:comp:2005}, systematic resampling can perform worse than multinomial in certain cases and stratified resampling is guaranteed to decrease the Monte Carlo variability in the particle samples, on average, relative to multinomial sampling.

%\begin{figure}
%\centering
%\includegraphics[width=1.0\textwidth]{PF-KD-normal}
%\caption{Displays 2.5\%/97.5\% quantiles of the filtered distributions over time of the unknown parameters (columns) for increasing number of particles (rows), color coded by resampling technique.  Normal priors on $\theta$ and the KDPF particle filter were used for all runs.  Plot axes are identical within columns, and areas shaded gray are outside of 95\% credible bounds calculated by taking the average quantiles of the four resampling techniques at $J = 20000$ particles.} \label{fig:resamp}
%\end{figure}

%\subsection{A note on when to resample}

%Regardless of resampling technique, it is inefficient to resample at every single iteration of the particle filter because particle weights may not be out of balance, and so resampling would be unnecessarily adding Monte Carlo variability to the particle sample.  As mentioned in Section \ref{sec:pf}, a measure of nonuniformity of particle weights called effective sample size was used to determine when to resample for all runs in our study \citep{Liu:Chen:Wong:reje:1998}.  This value can be interpreted as the number of independent particle samples and ranges between 1 and $J$, with $J$ corresponding to all particle weights being equal and 1 corresponding to one particle weight being 1 with the rest 0.  We use a threshold of 0.8 in our runs, meaning that if the number of independent samples is less than 80\% of the total number of particles, resampling is performed.  Other measures of nonuniformity include coefficient of variation and entropy, and future work is needed to determine which choice of nonuniformity measure and threshold level are optimal in certain cases.

%\section{Additional Unknown Parameters \label{sec:extend}}
%
%We now turn our focus to extending the analysis to include the $b_l$'s, $\varsigma_l$'s, and $\sigma_l$'s as unknown parameters (referred to as the extended analysis).  We consider data coming from just $L = 1$ stream and let $\theta = (\beta, \gamma, \nu, b, \varsigma, \sigma)'$, dropping the subscript $l$ for convenience.  Using the same simulated epidemic, data were simulated from equation \eqref{eqn:obs} with true values of $b$, $\varsigma$, and $\sigma$ set at $0.25$, $1$, and $0.001$, respectively.  Days at which data from the single stream were observed were randomly selected.
%
%The KDPF with tuning parameter $\delta$ set to 0.99 was run with $J = 20000$ particles, and stratified resampling was used with an effective sample size threshold of 0.8.  $x_0$ and $\theta$ were sampled from the prior density given by
%
%\[p\left(x_0,\theta\right) = p\left(\beta\right)p\left(\gamma\right)p\left(\nu\right)p\left(b\right)p\left(\varsigma\right)p\left(\sigma\right)p\left(i_0,s_0\right)\]
%
%\noindent where $p(i_0,s_0)$, $p(\beta)$, $p(\varsigma)$, and $p(\sigma)$ are defined the same way as in section \ref{sec:sim}.  $p(b)$, $p(\varsigma)$, and $p(\sigma)$ are defined by
%
%\begin{align*}
%\log(b) &\sim  N(-1.609, 0.3536^2) \\
%\log(\varsigma) &\sim N(-0.0114, 0.0771^2) \\
%\log(\sigma) &\sim N(-7.0516, 0.2803^2)
%\end{align*}
%
%As with the other three parameters, the prior distributions of $b$, $\varsigma$, and $\sigma$ are defined on the log scale so that they are constrained to be positive.  The choice of prior mean and standard deviation on the log scale were made such that random draws of $b$, $\varsigma$, and $\sigma$ on the original scale would be within $(0.1, 0.4)$, $(0.85, 1.15)$, and $(0.0005, 0.0015)$, respectively, with 95\% probability.  For comparison purposes, the KDPF with 20000 particles was also run assuming $b$, $\varsigma$, and $\sigma$ known (referred to as the original analysis).
%
%Figure \ref{fig:ext} shows 95\% credible intervals over time for marginal filtered distributions of the elements of $\theta$ and $x$ for both the extended (blue lines) and the original (red lines) analyses.  We notice from this figure that the lines appear choppier than in earlier plots, with flat periods followed by sharp changes in uncertainty.  This is due to the fact that data are coming from only one stream, leading to more time points where no data are available.  Gaps in the data lead to a lack of resampling of particles and causes more drastic shifts in the posterior distribution of $\theta$ once data arrive.  For example, a sharp shift in the distribution of $b$ occurs around $t = 30$ because of an influx of data following a period of no data and scarce resampling of particles.  Similarly, we notice a spike in the $s$, $i$, and $r$ curves around $t = 20$ for the same reason.
%
%Most obvious from Figure \ref{fig:ext} is that the intervals for $\beta$, $\gamma$, $\nu$, $s$, $i$, and $r$ are wider for the extended analysis than they are for the original.  This is expected due to the added uncertainty in $b$, $\varsigma$, and $\sigma$ in the extended model.  Nonetheless, we are still able to adequately estimate the epidemic curves even with less data available and imprecise knowledge of the parameters $b$, $\varsigma$, and $\sigma$ by increasing the number of particles used in the KDPF.
%
%\begin{figure}
%\centering
%\includegraphics[width=1.0\textwidth]{PF-ext-KD-stratified-normal-20000}
%\caption{Displays 2.5\%/97.5\% quantiles of the filtered distributions of the unknown states and parameters from the extended analysis (blue) over time compared with that from the original analysis (red).  $J = 20000$ particles for the KDPF were used with normal priors on $\theta$ and stratified resampling.  Tick marks are shown along the bottom of the plots for $\beta$ and $s$ at time points when data were observed (dark gray) and when particles were resampled (blue and red for the extended and original analyses, respectively).} \label{fig:ext}
%\end{figure}

\section{Discussion \label{sec:discussion}}

Presented here is a reanalysis of the model discussed in \citet{skvortsov2012monitoring} using the kernel density particle filter approach of \citet{Liu:West:comb:2001}. This filter provides a generic sequential inferential strategy that reduces particle degeneracy in many situations. It is simplistic yet appears to produce reasonable results in a large range of problems. It also allows for inference in more complicated situations, e.g. more unknown fixed parameters.

Other more sophisticated approaches exist as well, including an algorithm that incorporates an MCMC step to move the fixed parameters around \citep{Gilk:Berz:foll:2001,Stor:part:2002}. However, this approach requires more input by the practitioner, and the algorithm is no longer truly sequential since the computational effort increases with time.

%% this is for TeXShop to compile the main document
% !TEX root = pf.tex

Presented here is a strategy for estimating the current state of an epidemic along with other fixed parameters related to disease transmission and the generation of syndromic data. We use a compartmental epidemiological model of a disease outbreak and multivariate syndromic data for which any pattern of missingness is allowed. The epidemiological model is fused with syndromic information using a state-space modeling framework where the state evolution equation describes the evolution of the population proportion of susceptible and infected individuals over time, and the observation equation describes how the syndromic data are generated given the current state of the epidemic and fixed parameters. Both the state and observation equations are nonlinear with respect to the state and parameters, and they both incorporate stochasticity using a Gaussian distribution. In the state equation, this Gaussian distribution is truncated to restrict the elements of the state vector to be proportions with sum no greater than one, and we allow for negative correlation between the susceptible and infected populations that depends on the population size, contact rate of spread of disease, and recovery time from infection. We treat all fixed parameters in our model equally (i.e. either known or unknown) and incorporate an additional set of parameters into the observation equation taken from \citet{skvortsov2012monitoring} that adjusts the baseline level of incoming syndromic data.

We perform estimation of the state and unknown parameters at each day of a simulated epidemic using only data observed up through that day. Since the filtered distribution of the state and unknown parameters is mathematically intractable, we use particle filtering methods to estimate this distribution at each time point through a weighted sample approximation. Specifically, we use a version of the particle filter developed by \citet{Liu:West:comb:2001} that regenerates values of the fixed parameters with the intent of avoiding particle degeneracy, and we adapt this algorithm to regenerate fixed parameter values only at time points when resampling of particles is performed. Using uniform priors on the unknown, fixed parameters and systematic resampling, we show that this version of the particle filter performs more efficiently and is better at avoiding degeneracy at all stages of the epidemic when compared to the bootstrap filter and auxiliary particle filter.

We also implement our adapted version of the \citet{Liu:West:comb:2001} particle filter (which we call the KDPF) using log-normal priors on the unknown parameters and compare its performance to when uniform priors were used. Our results suggest that the uniform prior bounds for $\gamma$ and $\nu$ are too restrictive and have an unintended impact on the filtered distribution of the unknown parameters when used in conjunction with a logit transformation to allow the use of a normal density kernel when regenerating fixed parameter values. In addition, we show that implementing the KDPF with log-normal priors on the unknown parameters using multinomial resampling does not perform as well as when residual, stratified, or systematic resampling is used. Finally, we perform an extended analysis using the KDPF with stratified resampling on univariate data where all parameters of the model are regarded as unknown.

The KDPF provides a generic sequential inferential strategy that reduces particle degeneracy in many situations. This improved strategy allows for inference in more complicated situations, e.g. more unknown fixed parameters. Advanced techniques exist that are better at fighting particle degeneracy, but require more practitioner input.

For example, particle degeneracy could be combated within an SMC algorithm by incorporating a MCMC step to move the fixed parameters around \citep{Gilk:Berz:foll:2001,Stor:part:2002}. However, this would require the practitioner to define a MCMC algorithm in addition to the SMC algorithm. In addition to this requirement, the algorithm would no longer be truly sequential as the computational effort would increase with time.

Alternatively, if the practitioner is willing to modify their model, they can take advantage of a sufficient statistic structure \citep{Fear:mark:2002}, Rao-Blackwellization \citep{Douc:Gods:Andr:on:2000}, or both \citep{carvalho2010particle}. Possible modifications to the model in Section \ref{sec:model} to allow alternative strategies (such as these) include setting $\nu=1$, removing fixed parameters from $Q$, and eliminating the truncation in equation \eqref{eqn:state}. 

%  The \backmatter command formats the subsequent headings so that they
%  are in the journal style.  Please keep this command in your document
%  in this position, right after the final section of the main part of
%  the paper and right before the Acknowledgements, Supplementary Materials,
%  and References sections.

\backmatter

%  This section is optional.  Here is where you will want to cite
%  grants, people who helped with the paper, etc.  But keep it short!

%\section*{Acknowledgements}

%The authors thank Professor A. Sen for some helpful suggestions,
%Dr C. R. Rangarajan for a critical reading of the original version of the
%paper, and an anonymous referee for very useful comments that improved
%the presentation of the paper.\vspace*{-8pt}

%  If your paper refers to supplementary web material, then you MUST
%  include this section!!  See Instructions for Authors at the journal
%  website http://www.biometrics.tibs.org

%\section*{Supplementary Materials}

%Web Appendix A, referenced in Section~\ref{s:model}, is available with
%this paper at the Biometrics website on Wiley Online
%Library.\vspace*{-8pt}

%  Here, we create the bibliographic entries manually, following the
%  journal style.  If you use this method or use natbib, PLEASE PAY
%  CAREFUL ATTENTION TO THE BIBLIOGRAPHIC STYLE IN A RECENT ISSUE OF
%  THE JOURNAL AND FOLLOW IT!  Failure to follow stylistic conventions
%  just lengthens the time spend copyediting your paper and hence its
%  position in the publication queue should it be accepted.

%  We greatly prefer that you incorporate the references for your
%  article into the body of the article as we have done here
%  (you can use natbib or not as you choose) than use BiBTeX,
%  so that your article is self-contained in one file.
%  If you do use BiBTeX, please use the .bst file that comes with
%  the distribution.

%\begin{thebibliography}{}

%\bibitem{ } Cox, D. R. (1972). Regression models and life tables (with
%discussion).  \textit{Journal of the Royal Statistical Society, Series B}
%\textbf{34,} 187--200.

%\bibitem{ }  Hastie, T., Tibshirani, R., and Friedman, J. (2001). \textit{The
%Elements of Statistical Learning: Data Mining, Inference, and Prediction}.
%New York: Springer.

%\end{thebibliography}

\bibliographystyle{biom}
\bibliography{jarad}

%\appendix

%  To get the journal style of heading for an appendix, mimic the following.

%\section{}
%\subsection{Title of appendix}

%Put your short appendix here.  Remember, longer appendices are
%possible when presented as Supplementary Web Material.  Please
%review and follow the journal policy for this material, available
%under Instructions for Authors at \texttt{http://www.biometrics.tibs.org}.

\label{lastpage}

\end{document}
