\documentclass{article}

\usepackage{amsmath}
\usepackage{natbib}
\usepackage{color}
\usepackage{graphicx}
\usepackage[font={small,it}]{caption}

\graphicspath{{/Users/niemi/Dropbox/SIR_Particle_Filtering/Graphs/PF/}
                       {/Users/niemi/Dropbox/SIR_Particle_Filtering/doesnotexist/}
                       {/Users/Danny/Dropbox/SIR_Particle_Filtering/Graphs/PF/}
                       % Add your directories here
}

\newcommand{\jarad}[1]{{\color{red}JARAD: #1}}
\newcommand{\danny}[1]{{\color{blue}DANNY: #1}}

\begin{document}

\section{Introduction}

The rest of the article proceeds as follows. Section \ref{sec:model} contains a description of state-space models and the nonlinear dynamic model used in \citet{skvortsov2012monitoring}. Section \ref{sec:filtering} describes a variety of particle filtering methodologies including the kernel density approach of \cite{Liu:West:comb:2001}. In Section \ref{sec:apply}, we apply these methods to the \citet{skvortsov2012monitoring} model. In Section \ref{sec:extend}, we use the efficiency of these particle filtering methods to estimate a more complicated model. In Section \ref{sec:discussion}, we conclude with some additional particle filtering methods that may be helpful in certain model structures.

\section{State-space models \label{sec:model}}

State-space models are a general class of statistical models used for analysis of dynamic data in many fields including \jarad{references to uses of state-space models}. State space models are constructed using an observation equation, $y_t \stackrel{ind}{\sim} p_{y,t}(y_t|x_t,\theta)$, and a state evolution equation, $x_t \stackrel{ind}{\sim} p_{x,t}(x_t|x_{t-1},\theta)$ where $y_t$ is the observed response, $x_t$ is a latent, dynamic state, and $\theta$ is the unknown fixed parameter, all of which could be vectors. The distributions are assumed known conditional on the values of $\theta$ and $x_t$ in the observation equation and $\theta$ and $x_{t-1}$ in the evolution equation. Depending on whether the observations and the states are continuous or discrete, the distributions themselves may be continuous or discrete. The distributions are typically assumed to only vary with $x_t$ and $\theta$ and therefore the $t$ subscript is dropped.
%The particle filtering methodology discussed in Sections \ref{sec:filtering} and \ref{sec:discussion} apply equally well when the distributions depend on $t$ so long as they are known.
For simplicity, we will also drop the $x$ and $y$ subscript and instead let the arguments make clear which distribution we are referring to. Thus, the state-space model is
\begin{align*}
y_t &\stackrel{ind}{\sim} p(y_t|x_t,\theta) \\
x_t &\stackrel{ind}{\sim} p(x_t|x_{t-1},\theta).
\end{align*}

Special cases of these state-space models include hidden Markov models \jarad{HMM reference}, where the state $x_t$ has finite support, and dynamic linear models (DLMs) \citep{West:Harr:baye:1997}, where the distributions are both Gaussian whose means are linear functions of the states.

\subsection{Sequential estimation}

When data are collected sequentially, it is often of interest to determine the \emph{filtered distribution}, the distribution of the current state and parameter conditional on the data observed up to that time. This distribution describes all of the available information up to time $t$ about the current state of the system and any fixed parameters. It can be updated recursively using Bayes' rule:
\begin{equation}
p(x_t,\theta| y_{1:t}) \propto p(y_t|x_t,\theta)p(x_t,\theta|y_{1:t-1}) \label{eqn:filtered}
\end{equation}
where $y_{1:t} = (y_1,\ldots,y_t)$. Only in special cases can $p(x_t,\theta| y_{1:t})$ be evaluated analytically, e.g. in DLMs when $\theta$ is the observation variance \cite[Sec 4.3,][]{petris2009dynamic}. When analytical tractability is not present, we turn to numerical methods including deterministic versions, e.g. extended Kalman filter \jarad{ref} and the Gaussian sum filter \citep{Alsp:Sore:nonl:1972}, or Monte Carlo versions, i.e. particle filters.

\section{Particle filtering \label{sec:filtering}}

Particle filtering is a sequential Monte Carlo inferential technique based on sequential use of importance sampling. It aims to approximate equation \eqref{eqn:filtered} through a weighted Monte Carlo realization from this distribution, i.e.
\begin{equation}
p(x_t,\theta| y_{1:t}) \approx \sum_{j}^J w_t^{(j)} \delta_{(x_t^{(j)},\theta^{(j)})} \label{eqn:approx}
\end{equation}
where $w_t^{(j)},j=1,\ldots,J$ are the particle \emph{weights}, $(x_t,\theta)$ is the particle \emph{location}, and $\delta$ is a Dirac delta function. A variety of techniques have been developed to provide efficient approximations to equation \eqref{eqn:filtered} in the sense that with the same computation time a better approximation is achieved. We now introduce three fundamental particle filtering techniques: bootstrap filter, auxiliary particle filter, and the kernel density filter.

\subsection{Bootstrap Filter}

The first successful version of particle filtering is known as the bootstrap filter \citep{Gord:Salm:Smit:nove:1993}. Since this method and the auxiliary particle filter were developed for the situation when $\theta$ is known, we will (for the moment) drop $\theta$ from the notation. Suppose you have a current approximation as in equation \eqref{eqn:approx}. Then for each particle $j$,

\begin{enumerate}
\item Resample: sample an index $k\in \{1,\ldots,j,\ldots,J\}$ with probability $w_t^{(j)}$,
\item Propagate: sample $x_{t+1}^{(j)} \sim p\left(\left. x_{t+1}^{(k)}\right|x_t\right)$, and
\item Calculate weights: $\tilde{w}_{t+1}^{(j)} = p\left(y_{t+1}\left|x_{t+1}^{(j)}\right.\right)$.
\end{enumerate}

\noindent After weights for all particles have been calculated, these weights need to be renormalized via $w_{t+1}^{(j)} = \tilde{w}_{t+1}^{(j)}\left/ \sum_{l=1}^J \tilde{w}_{t+1}^{(l)} \right.$

This procedure provides an approximation to $p(x_t,\theta| y_{1:t})$ and therefore can be applied recursively provided an initial set of weights $w_0^{(j)}$ and locations $x_0^{(j)}$ for all $j$.

\subsection{Auxiliary Particle Filter}

A key problem that arises in implementing the bootstrap filter is that $w_t^{(j)}$ may be small for samples $x_t^{(j)}$ that are less likely given $y_t$. These samples will likely be eliminated during resampling, resulting in \emph{particle degeneracy}. One way to fight particle degeneracy is to increase the number of particles, $J$, used in the filter. A more efficient method, however, has been developed by \citet{Pitt:Shep:filt:1999} called the auxiliary particle filter.  The idea behind the auxiliary particle filter is to generate particles with higher likelihood given the data by "looking ahead" at $p(x_{t+1}|x_t^{(j)})$ prior to sampling. Given a weighted random sample of particles at time $t$, the auxiliary particle filter approximates $p(x_{t+1}|y_{1:t+1})$ by the following:

\begin{enumerate}
\item For each particle $j$, calculate a point estimate of $x_{t+1}^{(j)}$ called $\mu_{t+1}^{(j)}$.  %This is typically taken to be the mean of $p(x_{t+1}|x_t^{(j)})$.
\item Calculate auxiliary weights and renormalize
\[ \tilde{g}_{t+1}^{(j)} = w_t^{(j)} p(y_{t+1}|\mu_{t+1}^{(j)}) \qquad g_{t+1}^{(j)} = \tilde{g}_{t+1}^{(j)}\left/\sum_{l=1}^J \tilde{g}_{t+1}^l\right. \]
\item For each particle $j=1,\ldots,J$
	\begin{enumerate}
	\item Sample an index $k\in\{1,\ldots,j,\ldots,J\}$ with probability $g_{t+1}^{(j)}$.
	\item Propagate the particle $x_{t+1}^{(j)}\sim p\left(x_{t+1}|x_t^{(k)}\right)$.
	\item Calculate new weights and renormalize
\[\tilde{w}_{t+1}^{(j)} = \frac{p\left(y_{t+1}|x_{t+1}^{(j)}\right)}{p\left(y_{t+1}|\mu_{t+1}^{(k)}\right)} \qquad w_{t+1}^{(j)} = \tilde{w}_{t+1}^{(j)}\left/\sum_{l=1}^J \tilde{w}_{t+1}^l \right.\]
	\end{enumerate}
\end{enumerate}

The bootstrap filter and the auxiliary particle filter were constructed with the idea that all fixed parameters were known. In order to simultaneously estimate the states and fixed parameters using these methods it is necessary to incorporate the fixed parameters into the state with degenerate evolutions. Due to the resampling step, the number of unique values of the fixed parameters in out particle set will decrease over time, exacerbating the degeneracy problem \citep{Liu:West:comb:2001} \jarad{more references}.

\subsection{Kernel Density Particle Filter}

The particle filter introduced by \cite{Liu:West:comb:2001} is a general way of fighting this degeneracy problem by approximating the set of fixed parameter values by a kernel density estimate and then resampling from this approximation at each step in the filter. This enables us to refresh the distribution of $\theta$ to avoid degeneracy and require a lower number of particles to efficiently run the filter. We will refer to this filter as the kernel density particle filter.

We now reintroduce the fixed parameter $\theta$ into the notation. The kernel density particle filter provides an approximation to the filtered distribution via equation \eqref{eqn:approx}. To make the notation transparent, we introduce subscripts for our fixed parameters, e.g. $\theta_t^{(j)}$ represents the value for $\theta$ at time $t$ in particle $j$. This does not imply that $\theta$ is dynamic, but rather that the particle can have different values for $\theta$ throughout time.

Let $\bar{\theta}_t$ and $V_t$ be the weighted sample mean and weighted sample covariance matrix of the posterior sample $\theta_t^{(1)},\ldots,\theta_t^{(J)}$.  The kernel density particle filter also uses a tuning parameter $\delta$, the discount factor, and two derived quantities $h^2 = 1 - ((3\delta - 1)/2\delta)^2$ and $a^2 = 1 - h^2$ that determine how sharp the kernel density approximation is. Typically $\delta$ is taken to be between 0.95 and 0.99.

With the kernel density particle filter, we move to an approximation of $p(x_{t+1},\theta|y_{1:t+1})$ by the following steps:

\begin{enumerate}
\item For each particle $j$, calculate a point estimate of $\left(x_{t+1}^{(j)},\theta\right)$ given by $\left(\mu_{t+1}^{(j)},m_t^{(j)}\right)$ where
    \[
    \mu_{t+1}^{(j)} = E\left(x_{t+1}\left|x_t^{(j)},\theta_t^{(j)} \right.\right) \qquad
    m_t^{(j)} = a\theta_t^{(j)} + (1-a)\bar{\theta}_t
    \]
\item Calculate auxiliary weights and renormalize
\[ \tilde{g}_{t+1}^{(j)} = w_t^{(j)} p\left(y_{t+1}\left|\mu_{t+1}^{(j)},m_t^{(j)}\right.\right) \qquad g_{t+1}^{(j)} = \tilde{g}_{t+1}^{(j)}\left/ \sum_{l=1}^J \tilde{g}_{t+1}^l. \right. \]
\item For each particle $j=1,\ldots,J$
	\begin{enumerate}
	\item Sample an index $k\in\{1,\ldots,j,\ldots,J\}$ with probability $g_{t+1}^{(j)}$.
	\item Regenerate the fixed parameters
	\[ \theta_{t+1}^{(j)} \sim N\left( m_t^{(k)}, h^2V_t \right). \]
	\item Propagate the particle
	\[ x_{t+1}^{(j)} \sim p\left(x_{t+1}\left|x_t^{(k)},\theta_{t+1}^{(j)}\right.\right). \]
	\item Calculate weights and renormalize
	\[\tilde{w}_{t+1}^{(j)} = \frac{p\left(y_{t+1}\left|x_{t+1}^{(j)},\theta_{t+1}^{(j)}\right.\right)}{p\left(y_{t+1}\left|\mu_{t+1}^{(k)},m_t^{(k)}\right.\right)}
	\qquad
	w_{t+1}^{(j)} = \tilde{w}_{t+1}^{(j)}\left/\sum_{l=1}^J \tilde{w}_{t+1}^l \right.\]
	\end{enumerate}
\end{enumerate}

To use the kernel density particle filter with normal kernels it is necessary to parameterize the fixed parameters so that their support is on the real line. This is not a constraint, but rather a practical implementation detail. We typically use logarithms for parameters that have positive support and the logit function for parameters in the interval (0,1). A parameter $\psi$ bounded on the interval (a,b) can first be rebounded to (0,1) through $(\psi-a)/(b-a)$ and then the logit transformation can be applied.

\subsection{General advice}

Practical implementation of any particle filter involves a choice of resampling function and a decision about when resampling should be performed. Throughout our discussion we have implicitly used multinomial resampling, i.e. whenever we say `Sample an index $k\in\{1,\ldots,j,\ldots,J\}$ with probability $w$'. Alternatively, better resampling functions exist including stratified, residual, and systematic resampling. In addition, the frequency of resampling should be minimized to reduce Monte Carlo variability introduced during resampling. Typically a measure of the nonuniformity is used including effective sample size, entropy, and coefficient of variance. For a discussion of these topics please see \cite{Douc:Capp:Moul:comp:2005}.

\section{Particle filtering in epidemiological models \label{sec:apply}}

\subsection{Disease Dynamics}

As in \citet{skvortsov2012monitoring}, we consider the modified SIR model with stochastic fluctuations to describe the dynamics of the disease outbreak.  According to this model, we keep track of the number of susceptible, infected, and recovered individuals - denoted by $S$, $I$, and $R$, respectively - over the course of the epidemic.  We require $S + I + R = P$ where $P$ represents the total population size, and we define $s = S/P$, $i = I/P$, and $r = R/P$ so that $s + i + r = 1$.  The dynamics of the epidemic with respect to time, $t$, are then described by the following differential equations:

\begin{align}
\frac{ds}{dt} &= -\beta is^\nu + \epsilon_\beta \label{eqn:dsdt} \\
\frac{di}{dt} &= \beta is^\nu - \gamma i - \epsilon_\beta + \epsilon_\gamma \label{eqn:didt}
\end{align}

\noindent Here, $\beta$ represents the contact rate of spread of the disease, $\gamma$ is the recovery time, and $\nu \ge 0$ is the mixing intensity of the population.  $\epsilon_\beta$ and $\epsilon_\gamma$ are uncorrelated, Gaussian random variables with zero mean and standard deviations $\sigma_\beta$ and $\sigma_\gamma$, respectively.  Note that by definition, $r$ is completely specified once $s$ and $i$ are known.

As in \citet{skvortsov2012monitoring} once again, we will approximate the standard deviation of the noise terms $\sigma_\beta$ and $\sigma_\gamma$ by the scaling rule \[\sigma_\beta \approx \frac{\sqrt{\beta}}{P} \mbox{, } \sigma_\gamma \approx \frac{\sqrt{\gamma}}{P}\] and use Euler's method to approximate the solution of the differential equations given in \eqref{eqn:dsdt} and \eqref{eqn:didt}.  Letting $s_t$ and $i_t$ be the proportion of the population susceptible to disease and infected by disease at time $t$, respectively, the evolution of the epidemic from time $t$ to time $t + \tau$, $\tau > 0$, is approximated by

\begin{align}
s_{t+\tau} &= s_t - \tau\beta i_ts^\nu_t + \xi_s \label{eqn:s} \\
i_{t+\tau} &= i_t + \tau i_t(\beta s^\nu_t - \gamma) + \xi_i \label{eqn:i}
\end{align}

\noindent where $\xi_s$ and $\xi_i$ are independent, Gaussian random variables with zero mean and variances $(\beta + \gamma)\tau^2/P^2$ and $\beta\tau^2/P^2$, respectively.  This approximation is more accurate for small $\tau$.

\subsection{Model}

When monitoring the epidemic, the true $s$, $i$ and $r$ are unknown and regarded as hidden states of the model.  We instead observe measurements consisting of public health data, medical data such as hospital visits, and non-medical data such as absenteeism from work or school \danny{reference?}.  Thus, we characterize a state-space model of the epidemic by two equations: the observation equation and the state equation.  The observation equation specifies how the observed data depend on the state of the epidemic and other parameters.  We can observe data from multiple syndromes of a disease that arrive asynchronously in time.

Let $z_{n,k}$ represent data coming from syndrome $n$ at discrete-time index $k$ (i.e. at time $t_k$), with $k = 1,2,\ldots,T$ and $n = 1,2,\ldots,N$.  \citet{skvortsov2012monitoring} assumes a power law model relating observations to the proportion of infected individuals by $z_{n,k} = b_ni_{t_k}^{\varsigma_n} + \tau^*_n$ where $b_n \ge 0$ and $\varsigma_n$ are constants and $\tau^*_j$ is a random component. Since $z_{n,k}$ is constrained to be nonnegative, however, we let $y_{n,k} = \log z_{n,k}$ and model the log of the observations by

\begin{equation}
y_{n,k} = \log \left(b_ni_{t_k}^{\varsigma_n}\right) + \tau_n \qquad \tau_n \sim N\left(0,\sigma_n^2 / (b_ni_{t_k}^{\varsigma_n})^2\right) \label{eqn:obs}
\end{equation}

\noindent The variance of $\tau_n$ is obtained by assuming $\tau^*_n \sim N(0,\sigma_n^2)$ and applying the delta method to $z_{n,k}$ \danny{reference?}.

Next, we specify the state evolution equation.  The state equation explains how the unobserved states of the epidemic evolve over time.  Let $x_t = (i_t,s_t)'$ represent the state of the epidemic at time $t$, and let $\theta$ represent any unknown, fixed parameters in the model.  For the time being, we let $\theta = (\beta, \gamma, \nu)'$ and regard the $b_n$'s, $\varsigma_n$'s, and $\sigma_n$'s as known.  Letting $\tau > 0$ be small, the state evolution equation is given according to equations \eqref{eqn:s} and \eqref{eqn:i} by the transition density

\begin{equation}
p\left(x_{t+\tau}\left|x_t,\theta\right.\right) = N_{[0,1]\times[0,1]}\left(x_{t+\tau};f_\tau(x_t,\theta),Q_{\tau}(\theta)\right) \label{eqn:state}
\end{equation}

\noindent where

\[
f_\tau(x_t,\theta) = \left(
\begin{array}{c}
i_t + \tau i_t(\beta s^{\nu}_t - \gamma) \\
s_t - \tau\beta i_ts^{\nu}_t
\end{array}
\right)
\qquad
Q_\tau(\theta) = \left(
\begin{array}{ccccc}
(\beta + \gamma)\tau^2/P^2 & 0 \\
0 & \beta\tau^2/P^2
\end{array}
\right)
\]

\noindent and $N_{\Omega}(.;\mu,\Sigma)$ represents the pdf of the normal distribution with mean $\mu$ and covariance matrix $\Sigma$ truncated onto the set $\Omega$.  Notice that in equation \eqref{eqn:state} the marginal distributions of $s$ and $i$ have been truncated onto $[0,1]$ since they are proportions and it is possible from equations \eqref{eqn:s} and \eqref{eqn:i} to generate values outside of $[0,1]$.

Having formulated the observation and state equations, we can now express the likelihood of an observation $y_{j,k}$ given $x_t$ and $\theta$ by

\begin{equation}
p\left(y_{n,k}\left|x_{t_k},\theta\right.\right) = N\left(y_{n,k};\log(b_ni_{t_k}^{\varsigma_n}),\sigma_n^2 / (b_ni_{t_k}^{\varsigma_n})^2\right) \label{eqn:lik}
\end{equation}

Since $p(x_{t+\tau}|x_t,\theta)$ and $p(y_t|x_t,\theta)$ are nonlinear functions of $x_t$ and $\theta$, a closed-form solution to the posterior $p(x_t,\theta|y_{1:t})$ cannot be obtained.  Thus, we use particle filtering techniques described in section \ref{sec:filtering} to approximate $p(x_t,\theta|y_{1:t})$ for all $t$.

\subsection{A Note on Propagation}

Notice from equation \eqref{eqn:state} that the transition density needed to propagate $x_t$ in our model is a multivariate normal density with mean $f_\tau(x_t,\theta)$, and recall that that the mean function approximates the disease dynamics better for small $\tau$.  Thus, if the time between observations, $t_k +1 - t_k$, is large, then the mean of $p(x_{t_{k+1}}|x_{t_k}^{(j)}$ for each particle $j$ may be inaccurate.  To ensure that the transition density matches the actual disease dynamics more closely, we can choose $d$ to be an integer larger than 1, set $\tau = (t_{k+1} + 1 - t_k) / d$, and propagate $x_{t_k}^{(j)}$ forward $d$ times to generate $x_{t+1}^{(j)}$.

\subsection{Comparison of Particle Filters}

An epidemic was simulated from our model for $T = 125$ days for a population of size $P = 5000$.  True values of unknown parameters were set to $\beta = 0.2399$, $\gamma = 0.1066$, and $\nu = 1.2042$.  Values of known constants for $N = 4$ syndromes are given in table \ref{tab:true}.  Infection was introduced in 10 people in the population at day 0 (i.e. $i_0 = 10/5000$ and $s_0 = 4990/5000$).  Observations arrived daily from the different syndromes asynchronously, and the epidemic evolved over time according to equation \eqref{eqn:state} with $\tau = 0.2$ days.

\begin{table}[ht]
\begin{center}
\caption{Values of known constants in model.}
\label{tab:true}
\begin{tabular}{|cccc|}
\hline
$j$ & $b_j$ & $\varsigma_j$ & $\sigma_j$ \\
\hline
1 & 0.25 & 1.07 & 0.0012 \\
2 & 0.27 & 1.05 & 0.0008 \\
3 & 0.23 & 1.01 & 0.0010 \\
4 & 0.29 & 0.98 & 0.0011 \\
\hline
\end{tabular}
\end{center}
\end{table}

The bootstrap filter (BF), auxilliary particle filter (APF), and kernel density particle filter (KD) were each run using $n = 100, 1000, \mbox{ and } 10000$ particles to obtain weighted sample approximations to $p(x_{t_k},\theta|y_{1:k})$ for $k = 1,\ldots,T$.  $x_0$ and $\theta_0$ were sampled from the prior density given by

\[p\left(x_0,\theta\right) = p\left(\beta\right)p\left(\gamma\right)p\left(\nu\right)p\left(i_0,s_0\right)\] where
\begin{align*}
p(\beta) &= U_{[0.14,0.5]}(\beta) \\
p(\gamma) &= U_{[0.09,.143]}(\gamma) \\
p(\nu) &= U_{[0.95,1.3]}(\nu)
\end{align*}

\noindent where $U_{[a,b]}(.)$ is the pdf of a uniform distribution on the interval $[a,b]$.  $p(i_0,s_0)$ is the joint pdf of the random variables $i_0$ and $s_0$ defined by
\[i_0 = \frac{1}{b_n} \emph{e}^{\frac{y_{n,1}-\tau_n}{\varsigma_n}} \qquad s_0 = 1 - i_0 \]
where $\tau_n \sim N_{(y_{n,1} - \varsigma_n \log b_n,\infty]}(0,\sigma_n^2 / (b_n(0.001)^{\varsigma_n})^2)$ and $n$ corresponds to the syndrome from which the first observation came from.  The form of $i_0$ comes from rearranging equation \eqref{eqn:obs}, and the upper bound on the truncation of the distribution of $\tau_n$ ensures that $i_0$ is no greater than 1.

%Figure \ref{fig:quant} shows the mean and 95\% credible intervals of the proportion of infected individuals in the population over time using the quantiles of the weighted particle sample at each time point, and figure \ref{fig:hist} shows histograms of the unknown parameters after running the particle filters for $k = 25, 154$ days.  Notice in figure \ref{fig:quant} how the bootstrap and auxiliary particle filters degenerate early on in the epidemic because of particle deficiency, while the kernel density particle filter keeps a balanced distribution centered around the true proportion of the population infected.  In addition, it is possible to learn about the distribution of the unknown parameters over the course of the epidemic as is indicated in figure \ref{fig:hist}.  At day 25, the marginal posterior distribution of the contact rate parameter $\beta$ appears normal centered around the true value, and the marginal posterior distributions of $\gamma$ and $\nu$ appear more or less uniform, indicating that a lack of knowledge about these parameters within the first 25 days of the epidemic.  By the end of the epidemic, however, the distributions of all three parameters have shifted left.

%It is possible to forecast the epidemic curve after observing measurements up until time $t_k$ by propagating particles through the state equation \eqref{eqn:state} and setting $w_\tau = 0$.  This enables us to predict the timing and intensity of the peak of the epidemic early on.  Figure \ref{fig:pred} generates 95\% credible intervals of the predictive distribution $p(i_{t_m}|y_{1:k})$ for $m > k$ after running the kernel density particle filter for $k = 10, 15, \ldots, 35$ days.  Notice that the credible bounds all envelope the true proportion of infected individuals, and the timing of the peak of the epidemic appears to be consistent.

\begin{figure}[ht]
\centering
\begin{minipage}{1.0\linewidth}
\includegraphics[width=1.0\textwidth]{PF-states-100}
\end{minipage}
\begin{minipage}{1.0\linewidth}
\includegraphics[width=1.0\textwidth]{PF-states-1000}
\end{minipage}
\begin{minipage}{1.0\linewidth}
\includegraphics[width=1.0\textwidth]{PF-states-10000}
\end{minipage}
\caption{Each panel displays medians (solid colored lines) and 2.5\%/97.5\% quantiles (dashed colored lines) over time for the filtered distributions of the proportion of the population infected (left) and proportion of the population susceptible (right) using (top to bottom) $J = 100, 1000, \mbox{ and } 10000$ particles.  Different colors represent the different particle filter methods, and solid black lines shows the true simulated epidemic curves.} \label{fig:states}
\end{figure}

\begin{figure}[ht]
\centering
\begin{minipage}{1.1\linewidth}
\includegraphics[width=1.1\textwidth]{PF-params-100}
\end{minipage}
\begin{minipage}{1.1\linewidth}
\includegraphics[width=1.1\textwidth]{PF-params-1000}
\end{minipage}
\begin{minipage}{1.1\linewidth}
\includegraphics[width=1.1\textwidth]{PF-params-10000}
\end{minipage}
\caption{Each panel displays medians (solid colored lines) and 2.5\%/97.5\% quantiles (dashed colored lines) over time for the filtered distributions of the unknown parameters (columns) using (top to bottom) $J = 100, 1000, \mbox{ and } 10000$ particles.  Different colors represent the different particle filter methods, and horizontal solid black lines show the true value of the parameters used to simulate the data.} \label{fig:params}
\end{figure}

\begin{figure}[ht]
\centering
\begin{minipage}{1.0\linewidth}
\includegraphics[width=1.0\textwidth]{PF-states-med-100-100}
\end{minipage}
\begin{minipage}{1.1\linewidth}
\includegraphics[width=1.1\textwidth]{PF-params-med-100-100}
\end{minipage}
\caption{Each panel displays the median (solid colored lines) and 2.5\%/97.5\% quantiles (dashed colored lines) over time of $N = 100$ medians of the filtered distributions of the states (top) and unknown parameters (bottom).  Different colors represent the different particle filter methods using $J = 100$ particles.  In the top two panels, black solid and dashed lines represent the median and 2.5\%/97.5\% quantiles of the true simulated $i$ and $s$, and horizontal solid black lines in the bottom three panels show the true value of the parameters used to simulate the data.} \label{fig:meds}
\end{figure}

\section{Extending the model \label{sec:extend}}

\section{Discussion \label{sec:discussion}}

\bibliographystyle{plainnat}
\bibliography{jarad}

\end{document} 