% this is for TeXShop to compile the main document
% !TEX root = pf.tex

Presented here is a reanalysis of the model discussed in \citet{skvortsov2012monitoring} using the kernel density particular filter. This filter provides a generic sequential inferential strategy that reduces particle degeneracy in many situations. This improved strategy allows for inference in more complicated situations, e.g. more unknown fixed parameters. Advanced techniques exist that are better at fighting particle degeneracy, but require more practitioner input. 

For example, particle degeneracy can be combated within an SMC by incorporating an MCMC step to move the fixed parameters around \citep{Gilk:Berz:foll:2001,Stor:part:2002} which requires the practitioner to define an MCMC algorithm in addition to the SMC algorithm. In addition to this requirement, the algorithm is no longer truly sequential as the computational effort increases with time. 

Alternatively, if the practitioner is willing to modify their model, they can take advantage of a sufficient statistic structure \citep{Fear:mark:2002} or Rao-Blackwellization \citep{Douc:Gods:Andr:on:2000} or both \citep{carvalho2010particle}. Possible modifications to the model in Section \ref{sec:model} allowing alternative strategies include setting $\nu=1$, removing fixed parameters from $Q$, and eliminating the truncation in equation \eqref{eqn:state} as well as eliminating the variance dependence on the state in equation \eqref{eqn:obs}. 


