% this is for TeXShop to compile the main document
% !TEX root = pf.tex

Presented here is a strategy for simultaneous estimation of the current outbreak state and fixed parameters related to disease transmission using syndromic data. We introduce a stochastic epidemiological compartment model of a disease outbreak for data from syndromic surveillance that could possibly be multivariate and have any pattern of missingness. 
%The epidemiological model is fused with syndromic information using a state-space modeling framework where the state evolution equation describes the evolution of the population proportion of susceptible and infected individuals over time, and the observation equation describes how the syndromic data are generated given the current state of the epidemic and fixed parameters. Both the state and observation equations are nonlinear with respect to the state and fixed parameters, and they both incorporate stochasticity using a Gaussian distribution. In the state equation, this Gaussian distribution is truncated to restrict the elements of the state vector to be population proportions with sum no greater than one, and we allow for negative correlation between the susceptible and infected populations. This correlation depends on the population size, contact rate of spread of disease, and recovery time from infection. We treat all fixed parameters as either known or unknown and introduce an additional set of parameters into the observation equation taken from \citet{skvortsov2012monitoring} that adjusts the baseline level of incoming syndromic data.
%We perform estimation of the state and unknown fixed parameters at each day of a simulated epidemic using only data observed up through that day. 
We suggest the use of the kernel density particle filter \citep{Liu:West:comb:2001} using priors on fixed parameters that are bounded only by their support. We suggest the use of stratified (or residual) resampling when effective sample size has dropped markedly and regeneration of fixed parameters values should only occur when resampling is performed. We showed how this approach is capable of estimating a model with additional unknown fixed parameters. 
%Using uniform priors on the unknown fixed parameters together with systematic resampling, we show that this version of the particle filter performs more efficiently and is better at avoiding degeneracy at all stages of the epidemic for a fixed number of particles when compared to the bootstrap filter and auxiliary particle filter.

%We also implement our adapted version of the \citet{Liu:West:comb:2001} particle filter (which we call the KDPF) using log-normal priors on the unknown parameters and study sensitivity to prior specification by comparing results to a similar analysis using uniform priors. Our results suggest that the uniform prior bounds for $\gamma$ and $\nu$ taken from \citet{skvortsov2012monitoring} are too restrictive and have an unintended impact on the filtered distribution of the unknown parameters when used in conjunction with a logit transformation (to allow the use of a normal density kernel when regenerating fixed parameter values). In addition, we show that implementing the KDPF with log-normal priors on the unknown fixed parameters using multinomial resampling does not perform as well as when residual, stratified, or systematic resampling is used. Finally, we perform an extended analysis using the KDPF with stratified resampling on univariate data where all parameters of a single syndromic model are regarded as unknown.

The KDPF provides a generic sequential inferential strategy that reduces particle degeneracy in many situations. Advanced techniques exist that are better than the KDPF at fighting particle degeneracy, but require more practitioner input. For example, particle degeneracy could be combated within an SMC algorithm by incorporating a MCMC step to refresh fixed parameter values \citep{Gilk:Berz:foll:2001,Stor:part:2002}. However, this would require the practitioner to define a MCMC algorithm in addition to the SMC algorithm. In addition to this requirement, the algorithm would no longer be truly sequential as the computational effort increases with time. Alternatively, if the practitioner is willing to modify their model, they can take advantage of a sufficient statistic structure \citep{Fear:mark:2002}, Rao-Blackwellization \citep{Douc:Gods:Andr:on:2000}, or both \citep{carvalho2010particle}. Possible modifications to the model in Section \ref{sec:model} to allow alternative strategies (such as these) include setting $\nu=1$, removing fixed parameters from $Q$, and eliminating the truncation in equation \eqref{eqn:state}.

In this paper, we outline a strategy for real time tracking of a disease epidemic using data from syndromic surveillance, but this strategy can be applied to many other fields requiring on-line data analysis. We present improved particle filtering methods in general within the framework of sequential estimation of states and unknown fixed parameters in state-space models to inspire future work in epidemiological modeling and other scientific areas as well. 