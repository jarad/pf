% this is for TeXShop to compile the main document
% !TEX root = pf.tex

Presented here is a strategy for estimating the current state of an epidemic along with other fixed parameters related to disease transmission and the generation of syndromic data. We use a compartmental epidemiological model of a disease outbreak and multivariate syndromic data for which any pattern of missingness is allowed. The epidemiological model is fused with syndromic information using a state-space modeling framework where the state evolution equation describes the evolution of the proportions of a population susceptible to and infected by disease over time, and the observation equation describes how the syndromic data are generated given the current state of the epidemic and fixed parameters. Both the state and observation equations are nonlinear with respect to the state and parameters, and they both incorporate stochasticity using a Gaussian distribution. In the state equation, this Gaussian distribution is truncated to restrict the elements of the state vector to be proportions with sum no greater than one, and we allow for negative correlation between the susceptible and infected populations that depends on the population size, contact rate of spread of disease, and recovery time from infection. We treat all fixed parameters in our model equally (i.e. either known or unknown) and incorporate an additional set of parameters into the observation equation taken from \citet{skvortsov2012monitoring} that adjusts the baseline level of incoming syndromic data.

We perform estimation of the state and unknown parameters at each day of a simulated epidemic using only data observed up through that day. Since the filtered distribution of the state and unknown parameters is mathematically intractable, we use particle filtering methods to estimate this distribution at each time point through a weighted sample approximation. Specifically, we use a version of the particle filter developed by \citet{Liu:West:comb:2001} that regenerates values of the fixed parameters with the intent of avoiding particle degeneracy, and we adapt this algorithm to regenerate fixed parameter values only at time points when resampling of particles is performed. Using uniform priors on the unknown, fixed parameters and systematic resampling, we show that this version of the particle filter performs more efficiently and is better at avoiding degeneracy at all stages of the epidemic when compared to the bootstrap filter and auxiliary particle filter.

We also implement our adapted version of the \citet{Liu:West:comb:2001} particle filter (which we call the KDPF) using log-normal priors on the unknown parameters and compare its performance to when uniform priors were used. Our results suggest that the uniform prior bounds for $\gamma$ and $\nu$ are too restrictive and have an unintended impact on the filtered distribution of the unknown parameters when used in conjunction with a logit transformation to allow the use of a normal density kernel when regenerating fixed parameter values. In addition, we show that implementing the KDPF with log-normal priors on the unknown parameters using multinomial resampling does not perform as well as when residual, stratified, or systematic resampling is used. Finally, we perform an extended analysis using the KDPF with stratified resampling on univariate data where all parameters of the model are regarded as unknown.

The KDPF provides a generic sequential inferential strategy that reduces particle degeneracy in many situations. This improved strategy allows for inference in more complicated situations, e.g. more unknown fixed parameters. Advanced techniques exist that are better at fighting particle degeneracy, but require more practitioner input.

For example, particle degeneracy could be combated within an SMC algorithm by incorporating a MCMC step to move the fixed parameters around \citep{Gilk:Berz:foll:2001,Stor:part:2002}. However, this would require the practitioner to define a MCMC algorithm in addition to the SMC algorithm. In addition to this requirement, the algorithm would no longer be truly sequential as the computational effort would increase with time.

Alternatively, if the practitioner is willing to modify their model, they can take advantage of a sufficient statistic structure \citep{Fear:mark:2002}, Rao-Blackwellization \citep{Douc:Gods:Andr:on:2000}, or both \citep{carvalho2010particle}. Possible modifications to the model in Section \ref{sec:model} to allow alternative strategies (such as these) include setting $\nu=1$, removing fixed parameters from $Q$, and eliminating the truncation in equation \eqref{eqn:state}.