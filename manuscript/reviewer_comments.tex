\documentclass{article}

\usepackage{color}

\newcommand{\danny}[1]{{\color{blue} DANNY: #1}}

\begin{document}

\noindent {\bf Reviewers' comments:} \\

\noindent \emph{{\bf Handling Editor:}  I have read through the paper.  The epidemic model used in this paper seems to be essentially that of Skvortsov and Ristic (2012), which was published in Mathematical Biosciences.  However, the model is not described very clearly in the present paper, nor is it given any biological justification.  There is no justification for the choice of parameter values used in the simulated examples, except that many of them are the same as those used by Skvortsov and Ristic (2012).  Given that the epidemic model is essentially that of Skvortsov and Ristic (2012), the novelty of the paper seems to me to lie more in its statistical algorithms than in the application, so perhaps the paper is better suited to a statistics journal.  The reviewer has concerns about the scope of the present paper.  Taking all of this into account, I suggest that the authors are given an opportunity to revise their paper to address the above concerns and those of the reviewer but that no promises are made at this stage concerning the likely acceptance of such a revision.} \\

Thank you for the suggestion. We have expanded on the biological interpretation of model parameters in Section 4 as well as the choice of values for known parameters and priors for unknown parameters in Section 5. Although the novelty of our paper lies in the statistical algorithms, we hope to show that techniques such as the kernel density particle filter have a place in the analysis of biological data by applying these methods to the specific model used by Skvortsov and Ristic (2012). \\

\noindent \emph{{\bf Reviewer \#1:} The work in this paper concerns the application of Sequential Monte Carlo methods to the estimation of basic characteristics of an infectious epidemic outbreak in a Bayesian framework, and comparisons under different particle filters, resampling methods and prior distributions. The paper is very well written and its major contribution is the use of a kernel density particle filter in the context of epidemic modelling and estimation - which, to my knowledge, is novel.} \\

\noindent \emph{However, I feel that the work presented here is limited in the following ways:} \\

\noindent \emph{(1) All comparisons (of filters, resampling methods and priors) are solely based on a single simulated epidemic data set. For any reliable conclusions on the methodology used (and favored) by the authors, a thorough simulation study - or at least more extended simulations - should be carried out.} \\

We've expanded from analyzing a single simulated data set to 40 simulated data sets with different true parameter values. We compare the performance of the particle filter algorithms, resampling methods, and sensitivity to the tuning parameter $\Delta$ in the kernel density particle filter by analyzing the proportion of times that 95\% credible intervals for the states and parameters cover the true values used for simulation. Because our analysis has been expanded from 1 simulated data set to 40, much of the prose in Section 5 has been edited to accommodate conclusions drawn from multiple simulated epidemics. For instance, we've added a paragraph to the end of Section 5.3 that discusses the consistency of the results shown in Figure 2 across simulations. In addition, we've altered the analysis of resampling schemes in Section 5.5 to reflect an interpretation of Figure 4 (old Figure 6), which has been changed to display a comparison of the proportion of 95\% credible intervals that cover the true value used for simulation. Finally, we discuss sensitivity to the tuning parameter $\Delta$ in the KDPF in terms of its performance across the 40 simulations in Section 5.6.1.  \\

\noindent \emph{(2) The analysis and estimation is based on very informative priors on the model parameters, expressed as narrow uniform distributions or matched low-variance log-normal distributions. These are taken to be the same as in an earlier publication to facilitate comparisons, but raise several issues regarding the general applicability of the methods presented here: would the methods provide good inference if the priors were non-informative? For example, and very importantly, would they be applicable with real data where prior information could be limited?} \\

Any proper prior can be used in SMC algorithms, but priors that are more vague will require more particles to avoid degeneracy.
%If prior information about unknown quantities is limited to the point that priors are too vague to run an SMC algorithm with reasonable number of particles, common practice is to run an MCMC-type algorithm for a few time-points to inform a workable prior. While this extra step may be needed in specific cases, we argue that general applicability of SMC methods is not limited by this.
In response to this reviewer suggestion, we have altered our prior structure to put a prior on the basic reproductive number rather than the $S\to I$ transition rate where there is typically more prior information. \\

%In Section 5.2, we altered our log-normal priors to incorporate prior knowledge that may be available about the basic reproductive number. While these priors are informative, they are flexible in the event of model mis-specification and/or limited prior knowledge, allowing the data to overwhelm the prior and inform the posterior. \\

%Non-informative priors cannot be used with SMC methods, since we initialize an SMC algorithm by sampling from a proper prior distribution. However, we argue that general applicability is not limited by this. In Section 5.2, we use slightly different log-normal priors that are actually not that informative, and we show in Section 5.4 that these priors allow the data to overwhelm the prior and dictate the posterior distribution.


\noindent \emph{Also related to this point, one of the main conclusions of the presented work is that bounded priors influence estimation. This is a trivial point that should be expected when such restricted priors are used. The primary (and more general) issue here is that these priors are strongly informative.} \\

Although it may be a trivial point, we think it is worth making since these priors appear in the literature. We've altered Figure 3 within Section 5.4 to show that priors that may not seem informative can adversely influence estimation if they are not flexible. We've included a row showing samples from the KDPF run with log-normal priors, and we've added a $t = 0$ column displaying prior samples to demonstrate that the log-normal priors can be perceived as more informative than the uniform priors. Our point is that unbounded priors should be used so that - in the event that the prior and likelihood are vastly different - reasonable values of the particles can be found away from where the prior is concentrated. \\

%\danny{I think this needs to be reworded}

%In Figure 3 within Section 5.4, we show the prior samples at $t = 0$ and also have added a row of plots for the KDPF run using log-normal priors. Looking at the plots at $t = 0$, we explain in this section that the log-normal priors are actually more informative than the uniform priors. Our point is that unbounded priors should be used so that - in the event that the prior and likelihood are vastly different - reasonable values of the particles can be found away from where the prior is concentrated. \\

\noindent \emph{(3) The approach to epidemic modeling taken here only holds for Markovian models, where the state of the epidemic does not depend on the history of the outbreak (e.g. Eq. 3). This is quite restrictive and not realistic for many real-life epidemics. This issue has been addressed in the literature with the use of more general Bayesian modeling that extends to non-Markovian epidemic cases and includes non-exponential transitions between compartments. Can the approximations used in this paper allow for a more general framework?} \\

The general state-space model framework, described in Section 2, allows for non-Markovian structures, and any type of transition density can be specified in the state equation. We've added a paragraph in Section 2 to make this more clear. \\

\noindent \emph{(4) Although SMC methodology seems to offer a natural platform for real-time estimation in epidemics, other methods (related to sequential Bayesian analysis coupled with MCMC) have been considered in the literature. I feel that some comparison with such methodologies should be provided in this work. For example, most of the results suggest that the presented algorithms converge to the posterior distribution of the estimated parameters after around day 60 - which is after the epidemic has peaked. This provides a nice opportunity for comparing how MCMC would perform with data only from the first 60 days and then every, say, 5th day.} \\

We added Section 5.6.2  which compares MCMC vs KDPF (20000 particles, stratified resampling) using 95\% filtered credible intervals at time points 30, 60, 90, and 125. These results are comparable, and we explain that the KDPF also has the advantage of being much easier to implement and runs much faster. We apologize for not implementing the suggested every 5th day, but the increase in dimensionality of the state over time became too much of a computational burden to obtain the desired results in time. \\

\noindent \emph{Other comments:} \\

\noindent \emph{(5) The title of the paper  (``Estimation of a disease outbreak'') may be taken to suggest that the methodology presented here only applies to a very specific case. This is also related to my earlier comment (1). I think the title  it should imply more general applicability.} \\

We've changed the title to ``Comparison of the performance of particle filter algorithms applied to tracking of a disease epidemic.'' We've also updated the abstract accordingly. \\

\noindent \emph{(6) Page 1, line 28 (and elsewhere): the terminology "fixed parameters" contradicts the Bayesian approach and posterior (and prior) distributions.} \\

Most Bayesian statisticians, us included, believe in fixed parameters, but use a probability distribution to quantify our uncertainty about those fixed parameters. \\

\noindent \emph{(7) P1, l 31: the priors are referred to as "seemingly uninformative". However, these are quite informative priors.} \\

We agree. However, these priors are used in the literature, which is why we qualified uninformative with ``seemingly.'' We have added the quotes to seemingly in the text as well. \\

\noindent \emph{(8) The choice of the numerical value of the $\Delta$ parameter for the kernel smoother in the applications seems arbitrary. At the very least the sensitivity to changes of this value should be explored or discussed.} \\

We've run the KDPF with different values of $\Delta$ and discuss the sensitivity to this tuning parameter in an additional Section 5.6.1 that we've added. \\

\noindent \emph{(9) The various resampling techniques (e.g. p7, l14) should be briefly described in the paper for completeness.} \\

Included in the first paragraph of Section 3.4 are brief descriptions of stratified, residual, and systematic resampling, and we refer the reader to a paper that describes these methods and their asymptotic properties in more detail. If desired, we would be happy to include a more formal write-up of these resampling algorithms in an appendix. \\

\noindent \emph{(10) Section 3.5 could perhaps go to an appendix as I do not think that it contains information that is essential to the understanding of the paper.} \\

We think it is important for completeness of the paper to make the point that all of the particle filter algorithms are justified as the number of particles goes to infinity. This is related to comment (15) and explains why the difference between resampling algorithms becomes negligible for large $J$ and suggests that the filtered distributions are approaching the true posterior distributions. But we'd still be happy to put it in an appendix. \\

\noindent \emph{(11) P8, l47: The role of parameter $\nu$ must be explained and discussed. How does it control the mixing of the population?} \\

We've elaborated on this in Section 4.1.1. \\

\noindent \emph{(12) Is the covariance in Eq. (3) obvious? It should be explained - or a reference should be given.} \\

We've explained how this was calculated in Section 4.1.1. \\

\noindent \emph{(13) P9, l25: the physical interpretation of parameters $b$ and $\varsigma$ is not clear (to me). It should be discussed.} \\

In Section 4.1.2, we've elaborated more on where the power-law relationship between $\log y_{l,t}$ and $i$ comes from and the role these parameters play in the observation equation. \\

\noindent \emph{(14) Figure numbering (Fig 3, 4, 5).} \\

Fixed. \\

\noindent \emph{(15) Fig 6: The results here suggest that the difference between the 3 methods is almost negligible (for large J) except perhaps for parameter $\nu$. In any case, estimation of this parameter seems to be the most challenging - as perhaps expected given the lack of information on the mixing dynamics in the data. These issues should be discussed in the paper.} \\

The challenges of estimating $\nu$ is discussed in Section 5.3 as well as Section 6. Now that we have run with KDPF for each resampling scheme over multiple simulations, we've changed this figure (old Figure 6, new Figure 4) in Section 5.5 to show plots of the probability of coverage of the true values of parameters over all simulations. The conclusions drawn from this figure are essentially the same as the old Figure 6 and are now supported by evidence from multiple simulations and particle filter runs. \\

\noindent \emph{(16) P15, l22 (and Fig 6): How was the true posterior approximated here?} \\

In the old Figure 6, the true posterior was approximated by taking the average quantiles among the resampling schemes at the maximum number of particles used. Our new Figure 4 no longer displays an aggregated approximation to the true posterior. \\

%\danny{We don't have anything like this anymore?}

\noindent \emph{(17) Fig 7: The 95\% intervals suggest that estimation may not be converging for some parameters (e.g. $\varsigma$), and is marginally biased for some others (e.g. $\eta$). Again, this should be discussed.} \\

In Section 6, we discuss that because of the added uncertainty in the extended analysis, credible intervals for some parameters may not shorten much relative to the priors we've chosen, as is the case with $\nu$ and $\varsigma$ in particular. We cannot assess bias in the parameter estimates from one simulation. Credible intervals that are shifted from the true parameter value may be random results, as the credible intervals should cover the truth only 95\% of the time, or could be because not enough particles have been used in the KDPF. \\

%\danny{With dropping these figures, it seems like we are hiding things to me.}

\end{document}
