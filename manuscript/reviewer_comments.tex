\documentclass{article}

\begin{document}

\noindent {\bf Reviewers' comments:} \\

\noindent \emph{{\bf Handling Editor:}  I have read through the paper.  The epidemic model used in this paper seems to be essentially that of Skvortsov and Ristic (2012), which was published in Mathematical Biosciences.  However, the model is not described very clearly in the present paper, nor is it given any biological justification.  There is no justification for the choice of parameter values used in the simulated examples, except that many of them are the same as those used by Skvortsov and Ristic (2012).  Given that the epidemic model is essentially that of Skvortsov and Ristic (2012), the novelty of the paper seems to me to lie more in its statistical algorithms than in the application, so perhaps the paper is better suited to a statistics journal.  The reviewer has concerns about the scope of the present paper.  Taking all of this into account, I suggest that the authors are given an opportunity to revise their paper to address the above concerns and those of the reviewer but that no promises are made at this stage concerning the likely acceptance of such a revision.} \\

\noindent \emph{{\bf Reviewer \#1:} The work in this paper concerns the application of Sequential Monte Carlo methods to the estimation of basic characteristics of an infectious epidemic outbreak in a Bayesian framework, and comparisons under different particle filters, resampling methods and prior distributions. The paper is very well written and its major contribution is the use of a kernel density particle filter in the context of epidemic modelling and estimation - which, to my knowledge, is novel.} \\

\noindent \emph{However, I feel that the work presented here is limited in the following ways:} \\

\noindent \emph{(1) All comparisons (of filters, resampling methods and priors) are solely based on a single simulated epidemic data set. For any reliable conclusions on the methodology used (and favored) by the authors, a thorough simulation study - or at least more extended simulations - should be carried out.} \\

We've expanded from analyzing a single simulated data set to 20 simulated data sets, each with different true values of the model parameters. We compare the performance of the particle filter, resampling methods, and priors by analyzing the proportion of times 50\% credible intervals for the fixed parameters and states cover the true values used for simulation. \\

\noindent \emph{(2) The analysis and estimation is based on very informative priors on the model parameters, expressed as narrow uniform distributions or matched low-variance log-normal distributions. These are taken to be the same as in an earlier publication to facilitate comparisons, but raise several issues regarding the general applicability of the methods presented here: would the methods provide good inference if the priors were non-informative? For example, and very importantly, would they be applicable with real data where prior information could be limited?} \\

Non-informative priors cannot be used with SMC methods, since we initialize an SMC algorithm by sampling from a proper prior distribution. However, we argue that general applicability is not limited by this, as the priors used in our paper are actually not that informative and allow the data to overwhelm the prior and dictate the posterior distribution. \\

\noindent \emph{Also related to this point, one of the main conclusions of the presented work is that bounded priors influence estimation. This is a trivial point that should be expected when such restricted priors are used. The primary (and more general) issue here is that these priors are strongly informative.} \\

The bounded priors wouldn't influence estimation if they were chosen to be wider. Our point is that unbounded priors should be used so that - in the event that the prior and likelihood are vastly different - reasonable values of the particles can be found away from where the prior is concentrated. \\

\noindent \emph{(3) The approach to epidemic modeling taken here only holds for Markovian models, where the state of the epidemic does not depend on the history of the outbreak (e.g. Eq. 3). This is quite restrictive and not realistic for many real-life epidemics. This issue has been addressed in the literature with the use of more general Bayesian modelling that extends to non-Markovian epidemic cases and includes non-exponential transitions between compartments. Can the approximations used in this paper allow for a more general framework?} \\

The general state-space model framework described in Section 2 does in fact allow for non-Markovian structures, and any type of transition density can be specified in the state equation. We've added a paragraph in Section 2 to make this more clear. \\

\noindent \emph{(4) Although SMC methodology seems to offer a natural platform for real-time estimation in epidemics, other methods (related to sequential Bayesian analysis coupled with MCMC) have been considered in the literature. I feel that some comparison with such methodologies should be provided in this work. For example, most of the results suggest that the presented algorithms converge to the posterior distribution of the estimated parameters after around day 60 - which is after the epidemic has peaked. This provides a nice opportunity for comparing how MCMC would perform with data only from the first 60 days and then every, say, 5th day.} \\

We compare 95\% credible intervals of MCMC samples conditional the first 30, 60, and 125 data points with corresponding intervals from filtered distribution at those time points using the KD particle filter using 20000 particles and stratified resampling. \\

\noindent \emph{Other comments:} \\

\noindent \emph{(5) The title of the paper  ("Estimation of a disease outbreak .") may be taken to suggest that the methodology presented here only applies to a very specific case. This is also related to my earlier comment (1). I think the title  it should imply more general applicability.} \\

\noindent \emph{(6) Page 1, line 28 (and elsewhere): the terminology "fixed parameters" contradicts the Bayesian approach and posterior (and prior) distributions.} \\

\noindent \emph{(7) P1, l 31: the priors are referred to as "seemingly uninformative". However, these are quite informative priors.} \\

\noindent \emph{(8) The choice of the numerical value of the <DELTA> parameter for the kernel smoother in the applications seems arbitrary. At the very least the sensitivity to changes of this value should be explored or discussed.} \\

\noindent \emph{(9) The various resampling techniques (e.g. p7, l14) should be briefly described in the paper for completeness.} \\

Included in the first paragraph of Section 3.4 are brief descriptions of stratified, residual, and systematic resampling, and we refer the reader to a paper that describes these methods and their asymptotic properties in more detail. If desired, we could include a more formal write-up of these resampling algorithms in an appendix, but we decided against including too much detail in the main body of the paper because we did not think it essential to the main points of the paper. \\

\noindent \emph{(10) Section 3.5 could perhaps go to an appendix as I do not think that it contains information that is essential to the understanding of the paper.} \\

\noindent \emph{(11) P8, l47: The role of parameter $\nu$ must be explained and discussed. How does it control the mixing of the population?} \\

We've elaborated on this in Section 4.1.1. \\

\noindent \emph{(12) Is the covariance in Eq. (3) obvious? It should be explained - or a reference should be given.} \\

We've explained how this was calculated in Section 4.1.1. \\

\noindent \emph{(13) P9, l25: the physical interpretation of parameters $b$ and $\varsigma$ is not clear (to me). It should be discussed.} \\

In Section 4.1.2, we've elaborated more on where the power-law relationship between $\log y_{l,t}$ and $i$ comes from and the role these parameters play in the observation equation.

\noindent \emph{(14) Figure numbering (Fig 3, 4, 5).} \\

\noindent \emph{(15) Fig 6: The results here suggest that the difference between the 3 methods is almost negligible (for large J) except perhaps for parameter <nu>. In any case, estimation of this parameter seems to be the most challenging - as perhaps expected given the lack of information on the mixing dynamics in the data. These issues should be discussed in the paper.} \\

\noindent \emph{(16) P15, l22 (and Fig 6): How was the true posterior approximated here?} \\

\noindent \emph{(17) Fig 7: The 95\% intervals suggest that estimation may not be converging for some parameters (e.g. <zeta>), and is marginally biased for some others (e.g. <eta>). Again, this should be discussed.} \\

\end{document}
