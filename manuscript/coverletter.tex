% Cover letter using letter.sty
\documentclass{letter} % Uses 10pt
% Use \documentstyle[newcent]{letter} for New Century Schoolbook postscript font
% the following commands control the margins:
\topmargin=-1in    % Make letterhead start about 1 inch from top of page
\textheight=8in  % text height can be bigger for a longer letter
\oddsidemargin=0pt % leftmargin is 1 inch
\textwidth=6.5in   % textwidth of 6.5in leaves 1 inch for right margin

\begin{document}

\signature{Daniel M. Sheinson, Jarad Niemi, and Wendy Meiring}           % name for signature
\longindentation=0pt                       % needed to get closing flush left
\let\raggedleft\raggedright                % needed to get date flush left

\begin{letter}{Editor-in-Chief \\
The Wallace H. Coulter Dept. of Biomedical, Engineering \\
Georgia Tech and Emory University \\
313 Ferst Drive \\
Atlanta, GA 30332-0535, USA
}

\opening{Dear Editor,}

\noindent The attached submission presents particle filtering methodology for tracking and prediction of a disease outbreak using a compartmental epidemiological model and data from syndromic surveillance systems. Our paper was initially motivated as a reaction to "Monitoring and prediction of an epidemic outbreak using syndromic observations" (\emph{Mathematical Biosciences}, 2012) by Skvortsov and Ristic, as they present a strategy for a bio-surveillance system using a similar model of disease transmission and the bootstrap filter, the most basic particle filter that was developed by Gordon, Salmond, and Smith (1993). In our paper, we introduce more recent developments in the sequential Monte Carlo methodology that can perform more efficiently, including an algorithm developed by Liu and West (2001) which we refer to as the kernel density particle filter.

Our goal is to expose a wider audience to improved sequential Monte Carlo methods that can be more effective and efficient in the context of not only syndromic surveillance, but in modeling other biological processes as well. We therefore present these methods in general within the framework of sequential estimation of states and unknown parameters in state-space models, noting their flexibility and applicability to a wide array of fields. We then specify a compartmental epidemiological model that is similar to but not exactly the same as the one presented by Skvortsov and Ristic and compare the performance of how well the bootstrap filter, auxiliary particle filter (Pitt and Shepherd, 1999), and kernel density particle filter estimate the state of the epidemic and unknown parameters using data simulated from our model. In addition, we show the negative impact that bounded priors on the unknown fixed parameters, such as the uniform priors used by Skvortsov and Ristic, can have on estimation using the kernel density particle filter.

We also discuss some details concerning practical implementation of particle filters, namely choice of resampling scheme and when to resample particles, and demonstrate the shortcomings of multinomial resampling relative to other techniques. We conclude by mentioning some more sophisticated particle filtering methods that can be run under certain model settings to make the algorithm more efficient. Our aim here is to broaden awareness of the depth of methodology that has been developed for particle filtering to extend the possibilities for mathematical modeling in the biological sciences.

\closing{Sincerely yours,}

\end{letter}

\end{document}






