\documentclass[ucsb,pstat,12pt]{ucletter}
\name{Jarad Niemi}
\telephone{(805)~699-6572}
\email{niemi@pstat.ucsb.edu}
\signature{\vspace{-1.65cm}\hspace{-0cm}\includegraphics{JBN-color}\\
  \vspace{-0.2cm}Jarad Niemi\\Assistant Professor\\Department of Statistics \& Applied Probability}
% Use \documentstyle[newcent]{letter} for New Century Schoolbook postscript font
% the following commands control the margins:
\topmargin=-1in    % Make letterhead start about 1 inch from top of page
\textheight=8in  % text height can be bigger for a longer letter
\oddsidemargin=0pt % leftmargin is 1 inch
\textwidth=6.5in   % textwidth of 6.5in leaves 1 inch for right margin

\begin{document}

\signature{Daniel M. Sheinson, Jarad Niemi, and Wendy Meiring}           % name for signature
\longindentation=0pt                       % needed to get closing flush left
\let\raggedleft\raggedright                % needed to get date flush left

\begin{letter}{Dr. Eberhard O. Voit, Editor-in-Chief \\
The Wallace H. Coulter Dept. of Biomedical, Engineering \\
Georgia Tech and Emory University \\
313 Ferst Drive \\
Atlanta, GA 30332-0535, USA
}

\opening{Dear Dr. Voit,}

\noindent The attached submission presents particle filtering methodology for tracking and prediction of a disease outbreak using a compartmental epidemiological model and data from syndromic surveillance systems. Our paper was initially motivated by ``Monitoring and prediction of an epidemic outbreak using syndromic observations'' (\emph{Mathematical Biosciences}, 2012) by Alex Skvortsov and Branko Ristic, who present a strategy for a bio-surveillance system using a similar model of disease transmission and the bootstrap filter. The bootstrap filter, developed by Gordon, Salmond, and Smith in 1993, is the first successful particle filtering algorithm. It does, however, suffer from a severe degeneracy issue in the presence of unknown fixed parameters. In our paper, we introduce more recent developments in the sequential Monte Carlo methodology that produce more efficient algorithms, including one developed by Liu and West in 2001 that regenerates values of the fixed parameters to avoid degeneracy.

Our goal is to expose a wider audience to improved sequential Monte Carlo methods that can be more effective and efficient in the context of not only syndromic surveillance, but in modeling other biological processes as well. We therefore present these methods in general within the framework of sequential estimation of states and unknown parameters in state-space models, noting their flexibility and applicability to a wide array of fields. We then specify a compartmental epidemiological model that is similar to the one presented by Skvortsov and Ristic and compare the performance of several particle filtering algorithms in estimating the state of the epidemic and unknown fixed parameters using data simulated from our model. In addition, we show the negative impact that bounded priors on the unknown fixed parameters, such as the uniform priors used by Skvortsov and Ristic, can have on estimation and suggest the use of informative but unbounded priors.

We also discuss some details concerning practical implementation of particle filters, namely choice of resampling scheme and the frequency with which to resample particles, and demonstrate the shortcomings of multinomial resampling relative to other techniques. We conclude by mentioning some more sophisticated particle filters that can run more efficiently under certain model structures or with more coding effort. Our aim here is to broaden awareness of the depth of methodology that has been developed for particle filtering to extend the possibilities for mathematical modeling in the biological sciences.

\closing{Sincerely yours,}

\end{letter}

\end{document}






